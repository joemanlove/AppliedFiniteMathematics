\section{Combinations Involving Several Sets}

In this section, you will learn to:
\begin{enumerate}
    \item Count the number of items selected from more than one set.
    \item Count the number of items selected when there are restrictions on the selections.
\end{enumerate}

So far, we have solved the basic combination problem of \( r \) objects chosen from \( n \) different objects. Now we will consider certain variations of this problem.

\begin{example}
    How many five-person committees consisting of 2 faculty members and 3 students can be chosen from a total of 4 faculty members and 4 students?
\end{example}
\begin{solution}
    We list 4 faculty members and 4 students as follows:
    \[ F_1F_2F_3F_4S_1S_2S_3S_4 \]

    Since we want 5-person committees consisting of 2 faculty members and 3 students, we'll first form all possible two-faculty committees and all possible three-student committees. Clearly there are \( 4C2 = 6 \) two-faculty committees, and \( 4C3 = 4 \) three-student committees, we list them as follows:

    2-Faculty Committees
    \begin{itemize}
        \item \( F_1F_2 \)
        \item \( F_1F_3 \)
        \item \( F_1F_4 \)
        \item \( F_2F_3 \)
        \item \( F_2F_4 \)
        \item \( F_3F_4 \)
    \end{itemize}

    3-Student Committees
    \begin{itemize}
        \item \( S_1S_2S_3 \)
        \item \( S_1S_2S_4 \)
        \item \( S_1S_3S_4 \)
        \item \( S_2S_3S_4 \)
    \end{itemize}

    For every 2-faculty committee there are four 3-student committees that can be chosen to make a 5-person committee. If we choose \( F_1F_2 \) as our 2-faculty committee, then we can choose any of \( S_1S_2S_3 \), \( S_1S_2S_4 \), \( S_1S_3S_4 \), or \( S_2S_3S_4 \) as our 3-student committees. As a result, we get

    \[ \boxed{F_1F_2}S_1S_2S_3,~ \boxed{F_1F_2}S_1S_2S_4,~  \boxed{F_1F_2}S_1S_3S_4 ,~  \boxed{F_1F_2}S_2S_3S_4 \]

    Similarly, if we choose \( F_1F_3 \) as our 2-faculty committee, then, again, we can choose any of \( S_1S_2S_3 \), \( S_1S_2S_4 \), \( S_1S_3S_4 \), or \( S_2S_3S_4 \) as our 3-student committees.

    \[ \boxed{F_1F_3}S_1S_2S_3,~ \boxed{F_1F_3}S_1S_2S_4,~  \boxed{F_1F_3}S_1S_3S_4 ,~  \boxed{F_1F_3}S_2S_3S_4 \]

    And so on.

    Since there are six 2-faculty committees, and for every 2-faculty committee there are four 3-student committees, there are altogether \( 6 \cdot 4 = 24 \) five-person committees.

    In essence, we are applying the multiplication axiom to the different combinations.
\end{solution}

\begin{example}
    A club consists of 4 freshmen, 5 sophomores, 5 juniors, and 6 seniors. How many ways can a committee of 4 people be chosen that includes:
    \begin{enumerate}
        \item One student from each class?
        \item All juniors?
        \item Two freshmen and 2 seniors?
        \item No freshmen?
        \item At least three seniors?
    \end{enumerate}
\end{example}
\begin{solution}
    \begin{enumerate}
        \item Applying the multiplication axiom to the combinations involved, we get
              \[ (4C1)(5C1)(5C1)(6C1) = 600 \]
        \item We are choosing all 4 members from the 5 juniors, and none from the others.
              \[ 5C4 = 5 \]
        \item \( 4C2 \cdot 6C2 = 90 \)
        \item Since we don't want any freshmen on the committee, we need to choose all members from the remaining 16. That is
              \[ 16C4 = 1820 \]
        \item Of the 4 people on the committee, we want at least three seniors. This can be done in two ways. We could have three seniors, and one non-senior, or all four seniors.
              \[ (6C3)(14C1) + 6C4 = 295 \]
    \end{enumerate}
\end{solution}

\begin{example}
    How many five-letter word sequences consisting of 2 vowels and 3 consonants can be formed from the letters of the word INTRODUCE?
\end{example}
\begin{solution}
    First we select a group of five letters consisting of 2 vowels and 3 consonants. Since there are 4 vowels and 5 consonants, we have
    \[
        (4C2)(5C3)
    \]
    Since our next task is to make word sequences out of these letters, we multiply these by \(5!\):
    \[
        (4C2)(5C3)(5!) = 7200.
    \]
\end{solution}


\subsection{A Standard Deck of 52 Playing Cards}\label{subsection_standard_deck}

As in the previous example, many examples and homework problems in this book refer to a standard deck of 52 playing cards. Before we end this section, we take a minute to describe a standard deck of playing cards, as some readers may not be familiar with this.

A standard deck of 52 playing cards has 4 suits with 13 cards in each suit. The suits are diamonds (\diamonds), hearts (\hearts), spades (\spades), and clubs (\clubs). Each suit is associated with a color, either black (\spades, \clubs) or red (\diamonds, \hearts).

Each suit contains 13 denominations (or values) for cards: the nine numbers 2, 3, 4, ..., 10 and Jack(J), Queen (Q), King (K), Ace (A).

The Jack, Queen and King are called "face cards" because they have pictures on them. Therefore a standard deck has 12 face cards: (3 values JQK) x (4 suits {\hearts} {\diamonds} {\spades} {\clubs}). There are two Jacks in profile (\hearts, \spades) and two Jacks in full face (\diamonds, \clubs), the Jacks in profile are sometimes refered to as the one-eyed Jacks.

We can visualize the 52 cards by the following display

\begin{center}
    \begin{tabular}{l l l}
        \textbf{Suit}        & \textbf{Color} & \textbf{Values (Denominations)} \\
        {\diamonds} Diamonds & Red            & 2 3 4 5 6 7 8 9 10 J Q K A      \\
        {\hearts} Hearts     & Red            & 2 3 4 5 6 7 8 9 10 J Q K A      \\
        {\spades} Spades     & Black          & 2 3 4 5 6 7 8 9 10 J Q K A      \\
        {\clubs} Clubs       & Black          & 2 3 4 5 6 7 8 9 10 J Q K A      \\
    \end{tabular}
\end{center}


\begin{example}
    A standard deck of playing cards has 52 cards consisting of 4 suits with 13 cards in each. In how many different ways can a 5-card hand consisting of four cards of one suit and one of another be drawn?
\end{example}
\begin{solution}
    We will do the problem using the following steps.
    \begin{enumerate}
        \item Select a suit.
        \item Select four cards from this suit.
        \item Select another suit.
        \item Select a card from that suit.
    \end{enumerate}
    Applying the multiplication axiom, we have
    \begin{center}
        \begin{tabular}{lr}
            Ways of selecting the first suit         & $4C1$  \\
            Ways of selecting 4 cards from this suit & $13C4$ \\
            Ways of selecting the next suit          & $3C1$  \\
            Ways of selecting a card from that suit  & $13C1$ \\
        \end{tabular}
    \end{center}
    \[
        (4C1) \cdot (13C4) \cdot (3C1) \cdot (13C1) = 11,540.
    \]
\end{solution}
