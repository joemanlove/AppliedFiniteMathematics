\subsection{Permutations}
In this section, you will learn to:
\begin{enumerate}
    \item Count the number of possible permutations (ordered arrangement) of \( n \) items taken \( r \) at a time.
    \item Count the number of possible permutations when there are conditions imposed on the arrangements.
    \item Perform calculations using factorials.
\end{enumerate}

In Example \ref{example_tree_of_three_letter_sequences}, we were asked to find the word sequences formed by using the letters \( \{A, B, C\} \) if no letter is to be repeated. The tree diagram gave us the following six arrangements: \( ABC, ACB, BAC, BCA, CAB, \) and \( CBA \).

Arrangements like these, where order is important and no element is repeated, are called permutations.

\begin{definition}
    A \textbf{permutation} of a set of elements is an ordered arrangement where each element is used once.
\end{definition}

\begin{example}
    How many three-letter word sequences can be formed using the letters \{A, B, C, D\}?
\end{example}

\begin{solution}
    There are four choices for the first letter of our word, three choices for the second letter, and two choices for the third.
    \[
        \begin{array}{|c|c|c|}
            4 & 3 & 2 \\
            \hline
        \end{array}
    \]
    Applying the multiplication axiom, we get \( 4 \cdot 3 \cdot 2 = 24 \) different arrangements.
\end{solution}

\begin{example}
    How many permutations of the letters of the word ARTICLE have consonants in the first and last positions?
\end{example}

\begin{solution}
    In the word ARTICLE, there are 4 consonants.

    Since the first letter must be a consonant, we have four choices for the first position, and once we use up a consonant, there are only three consonants left for the last spot. We show as follows:
    \[
        \begin{array}{|c|c|c|c|c|c|c|}
            4 & ? & ? & ? & ? & ? & 3 \\
            \hline
        \end{array}
    \]
    Since there are no more restrictions, we can go ahead and make the choices for the rest of the positions.

    So far we have used up 2 letters, therefore, five remain. So for the next position there are five choices, for the position after that there are four choices, and so on. We get
    \[
        \begin{array}{|c|c|c|c|c|c|c|}
            4 & 5 & 4 & 3 & 2 & 1 & 3 \\
            \hline
        \end{array}
    \]
    So the total permutations are \( 4 \cdot 5 \cdot 4 \cdot 3 \cdot 2 \cdot 1 \cdot 3 = 1440 \).
\end{solution}

\begin{example}
    Given five letters \{A, B, C, D, E\}. Find the following:
    \begin{itemize}
        \item The number of four-letter word sequences.
        \item The number of three-letter word sequences.
        \item The number of two-letter word sequences.
    \end{itemize}
\end{example}

\begin{solution}
    The problem is easily solved by the multiplication axiom, and answers are as follows:
    \begin{itemize}
        \item The number of four-letter word sequences is \( 5 \cdot 4 \cdot 3 \cdot 2 = 120 \).
        \item The number of three-letter word sequences is \( 5 \cdot 4 \cdot 3 = 60 \).
        \item The number of two-letter word sequences is \( 5 \cdot 4 = 20 \).
    \end{itemize}
\end{solution}

We often encounter situations where we have a set of \( n \) objects and we are selecting \( r \) objects to form permutations. We refer to this as permutations of \( n \) objects taken \( r \) at a time, and we write it as \( nPr \).

Therefore, the above example can also be answered as listed below:
\begin{enumerate}
    \item The number of four-letter word sequences is \( 5P4 = 120 \).
    \item The number of three-letter word sequences is \( 5P3 = 60 \).
    \item The number of two-letter word sequences is \( 5P2 = 20 \).
\end{enumerate}

Before we give a formula for \( nPr \), we'd like to introduce a symbol that we will use a great deal in this as well as in the next chapter.

\begin{definition}
    The \textbf{factorial} of a natural number $n$ is given by
    \[
        n! = n\cdot(n-1)\cdot(n-2)\cdot\ldots\cdot3\cdot2\cdot1
    \]
    Also, we define $0! = 1$.

\end{definition}

\begin{definition}
    The number of permutations of $n$ objects taken $r$ at a time (with $n$ and $r$ natural numbers) denoted $nPr$ is given by either
    \[
        nPr = n\cdot(n-1)\cdot(n-2)\cdot\ldots\cdot(n-r+1)
    \]
    or
    \[
        nPr = \frac{n!}{(n-r)!}
    \]
\end{definition}

\begin{example}
    Compute the following using both formulas.
    \begin{enumerate}
        \item \( 6P3 \)
        \item \( 7P2 \)
    \end{enumerate}
\end{example}

\begin{solution}
    We will identify \( n \) and \( r \) in each case and solve using the formulas provided.
    \begin{enumerate}
        \item \( 6P3 = 6 \cdot 5 \cdot 4 = 120 \), or
              \[
                  6P3 = \frac{6!}{(6-3)!} = \frac{6!}{3!} = \frac{6 \cdot 5 \cdot 4 \cdot 3 \cdot 2 \cdot 1}{3 \cdot 2 \cdot 1} = 120
              \]

        \item \( 7P2 = 7 \cdot 6 = 42 \), or
              \[
                  7P2 = \frac{7!}{5!} = \frac{7 \cdot 6 \cdot 5 \cdot 4 \cdot 3 \cdot 2 \cdot 1}{5 \cdot 4 \cdot 3 \cdot 2 \cdot 1} = 42
              \]
    \end{enumerate}
\end{solution}

Next we consider some more permutation problems to get further insight into these concepts.

\begin{example}
    In how many different ways can 4 people be seated in a straight line if two of them insist on sitting next to each other?
\end{example}

\begin{solution}
    Let us suppose we have four people \( A \), \( B \), \( C \), and \( D \). Further suppose that \( A \) and \( B \) want to sit together. For the sake of argument, we glue \( A \) and \( B \) together and treat them as one person \( \boxed{AB} \).

    The four people are \( \boxed{AB} \) \( CD \). Since \( \boxed{AB} \) is treated as one person, we have the following possible arrangements.

    \[
        \boxed{AB}\ CD, \ \boxed{AB}\ DC, \ C\ \boxed{AB}\ D, \ D\ \boxed{AB}\ C, \ CD\ \boxed{AB}, \ DC\ \boxed{AB} \\
    \]

    Note that there are six more such permutations because \( A \) and \( B \) could also be glued together in the order \( \boxed{BA} \). And they are

    \[
        \boxed{BA}\ CD, \ \boxed{BA}\ DC, \ C\ \boxed{BA}\ D, \ D\ \boxed{BA}\ C, \ CD\ \boxed{BA}, \ DC\ \boxed{BA}
    \]
    So altogether there are \( 12 \) different permutations.

    Let us now do the problem using the multiplication axiom.

    After we glue two of the people together and treat them as one person, we can say we have only three people. The multiplication axiom tells us that three people can be seated in \( 3! \) ways. Since two people can be glued together \( 2! \) ways, there are \( 3! \times 2! = 12 \) different arrangements
\end{solution}

\begin{example}
    You have 4 math books and 5 history books to put on a shelf that has 5 slots. In how many ways can the books be shelved if the first three slots are filled with math books and the next two slots are filled with history books?
\end{example}

\begin{solution}
    We first do the problem using the multiplication axiom.

    Since the math books go in the first three slots, there are 4 choices for the first slot, 3 choices for the second and 2 choices for the third.

    The fourth slot requires a history book, and has five choices. Once that choice is made, there are 4 history books left, and therefore, 4 choices for the last slot. The choices are shown below.

    \[
        \begin{array}{|c|c|c|c|c|}
            4 & 3 & 2 & 5 & 4 \\
            \hline
        \end{array}
    \]

    Therefore, the number of permutations are \( 4 \cdot 3 \cdot 2 \cdot 5 \cdot 4 = 480 \).

    Alternately, we can see that \( 4 \cdot 3 \cdot 2 \) is really same as \( 4P3 \), and \( 5 \cdot 4 \) is \( 5P2 \).

    So the answer can be written as \( (4P3)(5P2) = 480 \).

    Clearly, this makes sense. For every permutation of three math books placed in the first three slots, there are \( 5P2 \) permutations of history books that can be placed in the last two slots. Hence the multiplication axiom applies, and we have the answer \( (4P3)(5P2) \).
\end{solution}
