\section{Systems of Linear Equations – Special Cases}
In this section you will learn to:
\begin{enumerate}
    \item Determine the linear systems that have no solution.
    \item Solve the linear systems that have infinitely many solutions.
\end{enumerate}
If we consider the intersection of two lines in a plane, three things can happen.
\begin{enumerate}
    \item The lines intersect in exactly one point. This is called an independent system.
    \item The lines are parallel, so they do not intersect. This is called an inconsistent system.
    \item The lines coincide; they intersect at infinitely many points. This is a dependent system.
\end{enumerate}
The figures below show all three cases:

\begin{minipage}{0.32\textwidth}
    \begin{tikzpicture}[scale=.65]
        % Draw axes
        \draw [<->,thick] (-3,0) -- (3,0) node [right] {$x$};
        \draw [<->,thick] (0,-3) -- (0,3) node [above] {$y$};

        % Draw lines
        \draw [<->,blue,dashed] (-1,3) -- (3,-1) node [right] {};
        \draw [<->,red] (-2,-2) -- (3,2) node [right] {};
    \end{tikzpicture}
\end{minipage}%
\begin{minipage}{0.32\textwidth}
    \begin{tikzpicture}[scale=.65]
        % Draw axes
        \draw [<->,thick] (-3,0) -- (3,0) node [right] {$x$};
        \draw [<->,thick] (0,-3) -- (0,3) node [above] {$y$};

        % Draw lines
        \draw [<->,blue,dashed] (-1,3) -- (3,-1) node [right] {};
        \draw [<->,red] (-2,3) -- (3,-2) node [right] {};
    \end{tikzpicture}
\end{minipage}%
\begin{minipage}{0.32\textwidth}
    \begin{tikzpicture}[scale=.65]
        % Draw axes
        \draw [<->,thick] (-3,0) -- (3,0) node [right] {$x$};
        \draw [<->,thick] (0,-3) -- (0,3) node [above] {$y$};

        % Draw lines
        \draw [<->,blue,dashed] (-2,3) -- (3,-2) node [right] {};
        \draw [<->,red] (-2,3) -- (3,-2) node [right] {};
    \end{tikzpicture}
\end{minipage}

Every system of equations has either one solution, no solution, or infinitely many solutions.

In the last section, we used the Gauss-Jordan method to solve systems that had exactly one solution.  In this section, we will determine the systems that have no solution, and solve the systems that have infinitely many solutions.

\begin{example}
    Solve the following system of equations using the Gauss-Jordan method:
    \begin{align*}
        x + y & = 7 \\
        x + y & = 9
    \end{align*}
\end{example}

\begin{solution}
    Let us use the Gauss-Jordan method to solve this system. The augmented matrix is
    \[
        \left[
            \begin{array}{cc|c}
                1 & 1 & 7 \\
                1 & 1 & 9
            \end{array}
            \right]
    \]

    If we multiply the first row by $-1$ and add it to the second row, we get
    \[
        \left[
            \begin{array}{cc|c}
                1 & 1 & 7 \\
                0 & 0 & 2
            \end{array}
            \right]
    \]

    Since $0$ cannot equal $2$, the last equation cannot be true for any choices of $x$ and $y$. Alternatively, it is clear that the two lines are parallel; therefore, they do not intersect.
\end{solution}

In the examples that follow, we are going to start using a calculator to row reduce the augmented matrix, in order to focus on understanding the answer rather than focusing on the process of carrying out the row operations.

\begin{example}
    Solve the following system of equations:
    \[
        \begin{aligned}
             & 2x + 3y - 4z = 7  \\
             & 3x + 4y - 2z = 9  \\
             & 5x + 7y - 6z = 20
        \end{aligned}
    \]
\end{example}

\begin{solution}
    We represent the system as an augmented matrix:
    \[
        \left[
            \begin{array}{ccc|c}
                2 & 3 & -4 & 7  \\
                3 & 4 & -2 & 9  \\
                5 & 7 & -6 & 20 \\
            \end{array}
            \right]
    \]

    By obtaining the reduced row-echelon form from a matrix calculator, we get:
    \[
        \left[\begin{array}{ccc|c}
                1 & 0 & 10 & 0 \\
                0 & 1 & -8 & 0 \\
                0 & 0 & 0  & 1 \\
            \end{array}\right]
    \]
    The bottom row implies \( 0x + 0y + 0z = 1 \), which is a contradiction. Thus, the system is inconsistent and has no solution.
\end{solution}


\begin{example}
    Solve the following system of equations:
    \[
        \begin{aligned}
             & x + y = 7 \\
             & x + y = 7
        \end{aligned}
    \]
\end{example}

\begin{solution}
    The problem asks for the intersection of two identical lines, meaning the lines coincide and intersect at an infinite number of points.

    A few intersection points are listed as follows: (3, 4), (5, 2), (-1, 8), (-6, 13), etc. However, when a system has an infinite number of solutions, the solution is often expressed in parametric form. This can be done by assigning an arbitrary constant, \( t \), to one of the variables and solving for the remaining variables. If we let \( y = t \), then \( x = 7 - t \). In other words, all ordered pairs of the form \( (7 - t, t) \) satisfy the given system of equations.

    Alternatively, solving with the Gauss-Jordan method, we obtain the reduced row-echelon form below, which includes a row of all zeros that can be ignored since it provides no additional information about the values of \( x \) and \( y \) that solve the system.

    \[
        \left[\begin{array}{cc|c}
                1 & 1 & 7 \\
                0 & 0 & 0 \\
            \end{array}\right]
    \]

    This leaves us with only one equation but two variables. Whenever there are more variables than equations, the solution must be expressed as a parametric solution in terms of an arbitrary constant, as shown above.

    Parametric Solution: \( x = 7 - t \), \( y = t \).
\end{solution}


\begin{example}
    Solve the following system of equations:
    \[
        \begin{aligned}
             & x + y + z = 2  \\
             & 2x + y - z = 3 \\
             & 3x + 2y = 5
        \end{aligned}
    \]
\end{example}

\begin{solution}
    The augmented matrix and the reduced row-echelon form are given below:
    \[
        \left[\begin{array}{ccc|c}
                1 & 1 & 1  & 2 \\
                2 & 1 & -1 & 3 \\
                3 & 2 & 0  & 5 \\
            \end{array}
            \right]
        \quad\rightarrow\quad
        \left[\begin{array}{ccc|c}
                1 & 0 & -2 & 1 \\
                0 & 1 & 3  & 1 \\
                0 & 0 & 0  & 0 \\
            \end{array}
            \right]
    \]

    Since the last equation dropped out, we are left with two equations and three variables. This means the system has an infinite number of solutions. We express those solutions in the parametric form by letting the last variable \( z \) equal the parameter \( t \).

    The first equation reads \( x - 2z = 1 \), therefore, \( x = 1 + 2z \).
    The second equation reads \( y + 3z = 1 \), therefore, \( y = 1 - 3z \).
    And now if we let \( z = t \), the parametric solution is expressed as follows:
    \[ \text{Parametric Solution:} \quad x = 1 + 2t, \quad y = 1 - 3t, \quad z = t. \]

    The reader should note that particular solutions, or specific solutions, to the system can be obtained by assigning values to the parameter \( t \). For example:
    \begin{itemize}
        \item If we let \( t = 2 \), we have the solution \( x = 5, y = -5, z = 2 \): \( (5, -5, 2) \).
        \item If we let \( t = 0 \), we have the solution \( x = 1, y = 1, z = 0 \): \( (1, 1, 0) \).
    \end{itemize}
\end{solution}

\begin{example}
    Solve the following system of equations:
    \[
        \begin{aligned}
             & x + 2y - 3z = 5   \\
             & 2x + 4y - 6z = 10 \\
             & 3x + 6y - 9z = 15
        \end{aligned}
    \]
\end{example}

\begin{solution}
    The reduced row-echelon form is given below:
    \[
        \left[
            \begin{array}{ccc|c}
                1 & 2 & -3 & 5 \\
                0 & 0 & 0  & 0 \\
                0 & 0 & 0  & 0 \\
            \end{array}
            \right]
    \]

    This time the last two equations drop out. We are left with one equation and three variables. Again, there are an infinite number of solutions. But this time the answer must be expressed in terms of two arbitrary constants.

    If we let \( z = t \) and \( y = s \), the first equation \( x + 2y - 3z = 5 \) results in \( x = 5 - 2s + 3t \). We rewrite the parametric solution as:
    \[ \text{Parametric Solution:} \quad x = 5 - 2s + 3t, \quad y = s, \quad z = t. \]
\end{solution}

\begin{summarybox}{Systems of Equations - Special Cases}
    \begin{enumerate}
        \item If any row of the reduced row-echelon form of the matrix gives a false statement such as \( 0 = 1 \), the system is inconsistent and has no solution.
        \item If the reduced row echelon form has fewer equations than the variables and the system is consistent, then the system has an infinite number of solutions. Remember the rows that contain all zeros are dropped.
              \begin{enumerate}
                  \item If a system has an infinite number of solutions, the solution must be expressed in the parametric form.
                  \item The number of arbitrary parameters equals the number of variables minus the number of equations.
              \end{enumerate}
    \end{enumerate}
\end{summarybox}