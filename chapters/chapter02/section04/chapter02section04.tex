\section{Inverse Matrices}
In this section you will learn to:
\begin{enumerate}
    \item Find the inverse of a matrix, if it exists.
    \item Use inverses to solve linear systems.
\end{enumerate}

In this section, we will learn to find the inverse of a matrix, if it exists. Later, we will use matrix inverses to solve linear systems.

\begin{definition}
    An \( n \times n \) matrix has an \textbf{inverse} if there exists a matrix \( B \) such that \( AB = BA = I_n \), where \( I_n \) is an \( n \times n \) identity matrix. The \textbf{inverse} of a matrix \( A \), if it exists, is denoted by the symbol \( A^{-1} \).
\end{definition}

\begin{example}
    Given matrices \( A \) and \( B \) below, verify that they are inverses.
    \[ A = \begin{bmatrix} 4 & 1 \\ 3 & 1 \end{bmatrix}, \quad B = \begin{bmatrix} 1 & -1 \\ -3 & 4 \end{bmatrix} \]
\end{example}
\begin{solution}
    The matrices are inverses if the product \( AB \) and \( BA \) both equal the identity matrix of dimension \( 2 \times 2 \), denoted as \( I_2 \):
    \[ AB = \begin{bmatrix} 4 & 1 \\ 3 & 1 \end{bmatrix} \begin{bmatrix} 1 & -1 \\ -3 & 4 \end{bmatrix} = \begin{bmatrix} 1 & 0 \\ 0 & 1 \end{bmatrix} = I_2 \]
    and
    \[ BA = \begin{bmatrix} 1 & -1 \\ -3 & 4 \end{bmatrix} \begin{bmatrix} 4 & 1 \\ 3 & 1 \end{bmatrix} = \begin{bmatrix} 1 & 0 \\ 0 & 1 \end{bmatrix} = I_2 \]
    Clearly, that is the case; therefore, the matrices \( A \) and \( B \) are inverses of each other.
\end{solution}

\begin{example}\label{calculate_2x2_inverse}
    Find the inverse of the matrix \( A = \begin{bmatrix} 3 & 1 \\ 5 & 2 \end{bmatrix} \).
\end{example}
\begin{solution}
    Suppose \( A \) has an inverse, and it is denoted as \( B = \begin{bmatrix} a & b \\ c & d \end{bmatrix} \).
    Then \( AB = I_2 \):
    \[ \begin{bmatrix} 3 & 1 \\ 5 & 2 \end{bmatrix} \begin{bmatrix} a & b \\ c & d \end{bmatrix} = \begin{bmatrix} 1 & 0 \\ 0 & 1 \end{bmatrix} \]
    After multiplying the matrices on the left side, we get the system:
    \[
        \begin{aligned}
            3a + c  & = 1 \\
            3b + d  & = 0 \\
            5a + 2c & = 0 \\
            5b + 2d & = 1
        \end{aligned}
    \]
    Solving this system, we find \( a = 2 \), \( b = -1 \), \( c = -5 \), and \( d = 3 \). Therefore, the inverse of matrix \( A \) is \( B = \begin{bmatrix} 2 & -1 \\ -5 & 3 \end{bmatrix} \).
\end{solution}

In this problem, finding the inverse of matrix \( A \) amounted to solving the system of equations:
\[
    \begin{aligned}
        3a + c  & = 1 \\
        3b + d  & = 0 \\
        5a + 2c & = 0 \\
        5b + 2d & = 1
    \end{aligned}
\]
Actually, it can be written as two systems, one with variables \( a \) and \( c \), and the other with \( b \) and \( d \). The augmented matrices for both are given below.
\[
    \left[
        \begin{array}{cc|c}
            3 & 1 & 1 \\
            5 & 2 & 0
        \end{array}
        \right]
    \quad \text{and} \quad
    \left[
        \begin{array}{cc|c}
            3 & 1 & 0 \\
            5 & 2 & 1
        \end{array}
        \right]
\]

As we look at the two augmented matrices, we notice that the coefficient matrix for both the matrices is the same. This implies the row operations of the Gauss-Jordan method will also be the same. A great deal of work can be saved if the two right-hand columns are grouped together to form one augmented matrix as below.
\[
    \left[
        \begin{array}{cc|cc}
            3 & 1 & 1 & 0 \\
            5 & 2 & 0 & 1
        \end{array}
        \right]
\]

And solving this system, we get
\[
    \left[
        \begin{array}{cc|cc}
            1 & 0 & 2  & -1 \\
            0 & 1 & -5 & 3
        \end{array}
        \right]
\]

The matrix on the right side of the vertical line is the \( A^{-1} \) matrix. What you just witnessed is no coincidence. This is the method that is often employed in finding the inverse of a matrix.

\begin{summarybox}
    \textbf{The Method for Finding the Inverse of a Matrix}
    \begin{enumerate}
        \item Write the augmented matrix $[ A | I_n ]$.
        \item Write the augmented matrix in step 1 in reduced row echelon form.
        \item If the reduced row echelon form in 2 is $[ I_n | B]$, then $B$ is the inverse of $A$.
        \item If the left side of the row reduced echelon is not an identity matrix, the inverse does not exist.
    \end{enumerate}

\end{summarybox}

\begin{example} \label{calculate_3x3_inverse}
    Given the matrix \( A \) below, find its inverse.
    \[ A = \begin{bmatrix}
            1 & -1 & 1 \\
            2 & 3  & 0 \\
            0 & -2 & 1
        \end{bmatrix} \]
\end{example}

\begin{solution}
    We write the augmented matrix as follows.
    \[ \left[
            \begin{array}{ccc|ccc}
                1 & -1 & 1 & 1 & 0 & 0 \\
                2 & 3  & 0 & 0 & 1 & 0 \\
                0 & -2 & 1 & 0 & 0 & 1
            \end{array}
            \right] \]

    We will reduce this matrix using the Gauss-Jordan method. Multiplying the first row by \(-2\) and adding it to the second row, we get
    \[ \left[
            \begin{array}{ccc|ccc}
                1 & -1 & 1  & 1  & 0 & 0 \\
                0 & 5  & -2 & -2 & 1 & 0 \\
                0 & -2 & 1  & 0  & 0 & 1
            \end{array}
            \right] \]

    If we swap the second and third rows, we get
    \[ \left[
            \begin{array}{ccc|ccc}
                1 & -1 & 1  & 1  & 0 & 0 \\
                0 & -2 & 1  & 0  & 0 & 1 \\
                0 & 5  & -2 & -2 & 1 & 0
            \end{array}
            \right] \]

    Divide the second row by \(-2\). The result is
    \[ \left[
            \begin{array}{ccc|ccc}
                1 & -1 & 1    & 1  & 0 & 0    \\
                0 & 1  & -1/2 & 0  & 0 & -1/2 \\
                0 & 5  & -2   & -2 & 1 & 0
            \end{array}
            \right] \]

    Let us do two operations here. \(1)\) Add the second row to the first. \(2)\) Add \(-5\) times the second row to the third. And we get
    \[ \left[
            \begin{array}{ccc|ccc}
                1 & 0 & 1/2  & 1  & 0 & -1/2 \\
                0 & 1 & -1/2 & 0  & 0 & -1/2 \\
                0 & 0 & 1/2  & -2 & 1 & 5/2
            \end{array}
            \right] \]

    Multiplying the third row by \(2\) results in
    \[ \left[
            \begin{array}{ccc|ccc}
                1 & 0 & 1/2  & 1  & 0 & -1/2 \\
                0 & 1 & -1/2 & 0  & 0 & -1/2 \\
                0 & 0 & 1    & -4 & 2 & 5
            \end{array}
            \right] \]

    Multiply the third row by \(1/2\) and add it to the second. Also, multiply the third row by \(-1/2\) and add it to the first.
    \[ \left[
            \begin{array}{ccc|ccc}
                1 & 0 & 0 & 3  & -1 & -3 \\
                0 & 1 & 0 & -2 & 1  & 2  \\
                0 & 0 & 1 & -4 & 2  & 5
            \end{array}
            \right] \]

    Therefore, the inverse of matrix \( A \) is \( A^{-1} = \begin{bmatrix}
        3  & -1 & -3 \\
        -2 & 1  & 2  \\
        -4 & 2  & 5
    \end{bmatrix} \).

    One should verify the result by multiplying the two matrices to see if the product does, indeed, equal the identity matrix.
\end{solution}
Now that we know how to find the inverse of a matrix, we will use inverses to solve systems of equations.  The method is analogous to solving a simple equation like the one below.
$$\frac{2}{3}x = 4$$

\begin{example}\label{solve_linear_equation}
    Solve the following equation:
    \[ x = 4 \]
\end{example}

\begin{solution}
    To solve the above equation, we multiply both sides of the equation by the multiplicative inverse of \( \frac{2}{3} \), which happens to be \( \frac{3}{2} \). We get
    \[ \frac{3}{2} \cdot \frac{2}{3} x = 4 \cdot \frac{3}{2} \]
    Hence,
    \[ x = 6. \]
\end{solution}

We use example \ref{solve_linear_equation} as an analogy to show how linear systems of the form \( AX = B \) are solved.
To solve a linear system, we first write the system in the matrix equation \( AX = B \), where \( A \) is the coefficient matrix, \( X \) is the matrix of variables, and \( B \) is the matrix of constant terms.
We then multiply both sides of this equation by the multiplicative inverse of the matrix \( A \).
Consider the following example.

\begin{example}
    Solve the following system
    \[
        \begin{aligned}
            3x + y  & = 3 \\
            5x + 2y & = 4
        \end{aligned}
    \]
\end{example}

\begin{solution}
    To solve the above equation, first we express the system as
    \[ AX = B \]
    where \( A \) is the coefficient matrix, and \( B \) is the matrix of constant terms. We get
    \[
        \left[
            \begin{array}{cc}
                3 & 1 \\
                5 & 2
            \end{array}
            \right]
        \left[
            \begin{array}{c}
                x \\
                y
            \end{array}
            \right]
        =
        \left[
            \begin{array}{c}
                3 \\
                4
            \end{array}
            \right]
    \]

    To solve this system, we multiply both sides of the matrix equation \( AX = B \) by \( A^{-1} \). Matrix multiplication is not commutative, so we need to multiply by \( A^{-1} \) on the left on both sides of the equation.

    Matrix \( A \) is the same matrix \( A \) whose inverse we found in Example \ref{calculate_2x2_inverse}, so \( A^{-1} = \left[
        \begin{array}{cc}
            2  & -1 \\
            -5 & 3
        \end{array}
        \right] \).

    Multiplying both sides by \( A^{-1} \), we get
    \[
        \left[
            \begin{array}{cc}
                2  & -1 \\
                -5 & 3
            \end{array}
            \right]
        \left[
            \begin{array}{cc}
                3 & 1 \\
                5 & 2
            \end{array}
            \right]
        \left[
            \begin{array}{c}
                x \\
                y
            \end{array}
            \right]
        =
        \left[
            \begin{array}{cc}
                2  & -1 \\
                -5 & 3
            \end{array}
            \right]
        \left[
            \begin{array}{c}
                3 \\
                4
            \end{array}
            \right]
    \]

    \[
        \left[
            \begin{array}{cc}
                1 & 0 \\
                0 & 1
            \end{array}
            \right]
        \left[
            \begin{array}{c}
                x \\
                y
            \end{array}
            \right]
        =
        \left[
            \begin{array}{c}
                2 \\
                -3
            \end{array}
            \right]
    \]

    \[
        \left[
            \begin{array}{c}
                x \\
                y
            \end{array}
            \right]
        =
        \left[
            \begin{array}{c}
                2 \\
                -3
            \end{array}
            \right]
    \]

    Therefore, \( x = 2 \), and \( y = -3 \).
\end{solution}

\begin{example}
    Solve the following system:
    \begin{align*}
        x - y + z & = 6 \\
        2x + 3y   & = 1 \\
        -2y + z   & = 5
    \end{align*}
\end{example}

\begin{solution}
    To solve the above equation, we write the system in matrix form \( AX = B \) as follows:
    \[ \left[
            \begin{array}{ccc}
                1 & -1 & 1 \\
                2 & 3  & 0 \\
                0 & -2 & 1
            \end{array}
            \right]
        \left[
            \begin{array}{c}
                x \\
                y \\
                z
            \end{array}
            \right] =
        \left[
            \begin{array}{c}
                6 \\
                1 \\
                5
            \end{array}
            \right] \]

    To solve this system, we need the inverse of \( A \). From Example \ref{calculate_3x3_inverse}, \( A^{-1} \) is given by
    \[ A^{-1} = \left[
            \begin{array}{ccc}
                3  & -1 & -3 \\
                -2 & 1  & 2  \\
                -4 & 2  & 5
            \end{array}
            \right] \]

    Multiplying both sides of the matrix equation \( AX = B \) on the left by \( A^{-1} \), we get
    \[ \left[
            \begin{array}{ccc}
                3  & -1 & -3 \\
                -2 & 1  & 2  \\
                -4 & 2  & 5
            \end{array}
            \right]
        \left[
            \begin{array}{c}
                6 \\
                1 \\
                5
            \end{array}
            \right] =
        \left[
            \begin{array}{c}
                x \\
                y \\
                z
            \end{array}
            \right] \]

    After multiplying the matrices, we get
    \[ \left[
            \begin{array}{ccc}
                1 & 0 & 0 \\
                0 & 1 & 0 \\
                0 & 0 & 1
            \end{array}
            \right]
        \left[
            \begin{array}{c}
                x \\
                y \\
                z
            \end{array}
            \right] =
        \left[
            \begin{array}{c}
                2  \\
                -1 \\
                3
            \end{array}
            \right] \]

    \[ \left[
            \begin{array}{c}
                x \\
                y \\
                z
            \end{array}
            \right] =
        \left[
            \begin{array}{c}
                2  \\
                -1 \\
                3
            \end{array}
            \right] \]

    Therefore, \( x = 2 \), \( y = -1 \), and \( z = 3 \).
\end{solution}

We remind the reader that not every system of equations can be solved by the matrix inverse method.  Although the Gauss-Jordan method works for every situation, the matrix inverse method works only in cases where the inverse of the square matrix exists.  In such cases the system has a unique solution.

\begin{summarybox}{Finding the Inverse of a Matrix}
    \begin{enumerate}
        \item Write the augmented matrix $[ A | I_n ]$.
        \item Write the augmented matrix in step 1 in reduced row echelon form.
        \item If the reduced row echelon form in step 2 is $[ I_n | B]$, then $B$ is the inverse of $A$.
        \item If the left side of the row reduced echelon is not an identity matrix, the inverse does not exist.
    \end{enumerate}
\end{summarybox}
\begin{summarybox}{Solving a System of Equations When a Unique Solution Exists}
    \begin{enumerate}
        \item Express the system in the matrix equation $AX = B$.
        \item To solve the equation $AX = B$, multiply both sides by $A^{-1}$:
              \[
                  AX = B
              \]
              \[
                  A^{-1}AX = A^{-1}B
              \]
              \[
                  I X = A^{-1}B \quad \text{where $I$ is the identity matrix}
              \]
    \end{enumerate}

\end{summarybox}