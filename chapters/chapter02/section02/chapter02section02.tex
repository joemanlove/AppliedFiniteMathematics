\section{Systems of Linear Equations; Gauss-Jordan Method}
In this section you will learn to
\begin{enumerate}
    \item Represent a system of linear equations as an augmented matrix
    \item Solve the system using elementary row operations.
\end{enumerate}
In this section, we learn to solve systems of linear equations using a process called the Gauss-Jordan method. The process begins by first expressing the system as a matrix, and then reducing it to an equivalent system by simple row operations. The process is continued until the solution is obvious from the matrix. The matrix that represents the system is called the augmented matrix, and the arithmetic manipulation that is used to move from a system to a reduced equivalent system is called a row operation.

\begin{example}
    Write the following system as an augmented matrix.
    \begin{align*}
        2x + 3y - 4z & = 5  \\
        3x + 4y - 5z & = -6 \\
        4x + 5y - 6z & = 7
    \end{align*}
\end{example}
\begin{solution}
    We express the above information in matrix form. Since a system is entirely determined by its coefficient matrix and by its matrix of constant terms, the augmented matrix will include only the coefficient matrix and the constant matrix. So the augmented matrix we get is as follows:
    \[
        \left[
            \begin{array}{ccc|c}
                2 & 3 & -4 & 5  \\
                3 & 4 & -5 & -6 \\
                4 & 5 & -6 & 7  \\
            \end{array}
            \right]
    \]
\end{solution}

In the last section, we expressed the system of equations as $AX = B$, where $A$ represented the coefficient matrix, and $B$ the matrix of constant terms. As an augmented matrix, we write the matrix as $[ A | B]$. It is clear that all of the information is maintained in this matrix form, and only the letters $x$, $y$, and $z$ are missing. A student may choose to write $x$, $y$, and $z$ on top of the first three columns to help ease the transition.

\begin{example}
    For the following augmented matrix, write the system of equations it represents.
    \[
        \left[
            \begin{array}{ccc|c}
                1 & 3 & -5 & 2  \\
                2 & 0 & -3 & -5 \\
                3 & 2 & -3 & -1
            \end{array}
            \right]
    \]
\end{example}
\begin{solution}
    The system is readily obtained as below.
    \begin{align*}
        x + 3y - 5z  & = 2  \\
        2x      - 3z & = -5 \\
        3x + 2y - 3z & = -1
    \end{align*}
\end{solution}
Once a system is expressed as an augmented matrix, the Gauss-Jordan method reduces the system into a series of equivalent systems by using the row operations. This row reduction continues until the system is expressed in what is called the reduced row echelon form. The reduced row echelon form of the coefficient matrix has 1's along the main diagonal and zeros elsewhere. The solution is readily obtained from this form.

