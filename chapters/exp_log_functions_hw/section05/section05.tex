\section{Log and Exponential Application Puzzles}

\begin{puzzle}
    An investment's value is rising at the rate of 5\% per year. The initial value of the investment is \$20,000 in 2016.
    \begin{enumerate}
        \item Write the function that gives the value of the investment as a function of time \( t \) in years after 2016.
        \item Find the value of the investment in 2028.
        \item When will the value be \$30,000?
    \end{enumerate}
\end{puzzle}

\begin{puzzle}
    The population of a city is increasing at the rate of 2.3\% per year, since the year 2000. Its population in 2010 was 137,000 people.
    \begin{enumerate}
        \item Find the population of the city in the year 2000.
    \end{enumerate}
\end{puzzle}

\begin{puzzle}
    The value of a piece of industrial equipment depreciates after it is purchased. Suppose that the depreciation follows an exponential decay model. The value of the equipment at the end of 8 years is \$30,000 and its value has been decreasing at the rate of 7.5\% per year.
    \begin{enumerate}
        \item Find the initial value of the equipment when it was purchased.
    \end{enumerate}
\end{puzzle}

\begin{puzzle}
    An investment has been losing money. Its value has been decreasing at the rate of 3.2\% per year. The initial value of the investment was \$75,000 in 2010.
    \begin{enumerate}
        \item Write the function that gives the value of the investment as a function of time \( t \) in years after 2010.
        \item If the investment's value continues to decrease at this rate, find the value of the investment in 2020.
    \end{enumerate}
\end{puzzle}

\begin{puzzle}
    A social media site has 275 members initially. The number of members has been increasing exponentially according to the function \( y = 275e^{0.1t} \), where \( t \) is the number of months since the site's initial launch.
    \begin{enumerate}
        \item How many months does it take until the site has 5000 members? State the answer to the nearest tenth of a month (1 decimal place).
    \end{enumerate}
\end{puzzle}

\begin{puzzle}
    A city has a population of 62000 people in the year 2000. Due to high unemployment, the city's population has been decreasing at the rate of 2\% per year. Using this model,
    \begin{enumerate}
        \item Find the population of this city in 2016.
    \end{enumerate}
\end{puzzle}

\begin{puzzle}
    A city has a population of 87,000 people in the year 2000. The city's population has been increasing at the rate of 1.5\% per year.
    \begin{enumerate}
        \item How many years does it take until the population reaches 100,000 people?
    \end{enumerate}
\end{puzzle}

\begin{puzzle}
    An investment of \$50,000 is increasing in value at the rate of 6.3\% per year.
    \begin{enumerate}
        \item How many years does it take until the investment is worth \$70,000?
    \end{enumerate}
\end{puzzle}

\begin{puzzle}
    A city has a population of 50,000 people in the year 2000. The city’s population increases at a constant percentage rate. Fifteen years later, in 2015, the population of this city was 70,000.
    \begin{enumerate}
        \item Find the annual percentage growth rate.
    \end{enumerate}
\end{puzzle}

\begin{puzzle}
    200 mg of a medication is administered to a patient. After 3 hours, only 100 mg remains in the bloodstream. Using an exponential decay model,
    \begin{enumerate}
        \item Find the hourly decay rate.
    \end{enumerate}
\end{puzzle}

\begin{puzzle}
    An investment is losing money at a constant percentage rate per year. The investment was initially worth \$25,000 but is worth only \$20,000 after 4 years.
    \begin{enumerate}
        \item Find the percentage rate at which the investment is losing value each year (that is, find the annual decay rate).
    \end{enumerate}
\end{puzzle}

\begin{puzzle}
    Using the information in the previous puzzle,
    \begin{enumerate}
        \item How many years does it take until the investment is worth only half of its initial value?
    \end{enumerate}
\end{puzzle}

