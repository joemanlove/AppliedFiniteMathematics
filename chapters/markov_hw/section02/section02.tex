\section{Regular Markov Chain Puzzles}

\begin{puzzle}
    Determine whether the following matrices are regular Markov chains.
    \begin{enumerate}
        \item \(
              \begin{bmatrix}
                  1  & 0  \\
                  .5 & .5 \\
              \end{bmatrix}
              \)
        \item \(
              \begin{bmatrix}
                  .6 & .4 \\
                  0  & 1  \\
              \end{bmatrix}
              \)
        \item \(
              \begin{bmatrix}
                  .2 & .4 & .4 \\
                  0  & 0  & 0  \\
              \end{bmatrix}
              \)
        \item \(
              \begin{bmatrix}
                  .6 & .4 & 0  \\
                  .3 & .2 & .5 \\
              \end{bmatrix}
              \)
    \end{enumerate}
\end{puzzle}

\begin{puzzle}
    Company I and Company II compete against each other, and the transition table for people switching from Company I to Company II is given below.

    \begin{center}
        \begin{tabular}{ll|cc}
                 &            & To        &            \\
                 &            & Company I & Company II \\
            \hline
            From & Company I  & .3        & .7         \\
                 & Company II & .8        & .2         \\
        \end{tabular}
    \end{center}

    \begin{enumerate}
        \item If the initial market share is 40\% for Company I and 60\% for Company II, what will the market share be after 3 steps?
        \item If this trend continues, what is the long range expectation for the market?
    \end{enumerate}
\end{puzzle}

\begin{puzzle}
    Suppose the transition table for the tennis player in Puzzle \ref{puzzle_markov_tennis} is as follows, where C denotes the cross-court shots and D denotes down-the-line shots.

    \begin{center}
        \begin{tabular}{ll|cc}
                     &   & Next &    \\
                     &   & C    & D  \\
            \hline
            Previous & C & .9   & .1 \\
                     & D & .7   & .3 \\
        \end{tabular}
    \end{center}

    \begin{enumerate}
        \item If the player hit the first shot cross-court, what is the probability he will hit the fourth shot
              cross-court?
        \item Determine the long term shot distribution
    \end{enumerate}
\end{puzzle}

\begin{puzzle}
    Professor Hay never orders eggs two days in a row, but if he orders tofu one day, then there is an equal probability that he will order tofu or eggs the next day. Find the following:
    \begin{enumerate}
        \item If Professor Hay had eggs on Monday, what is the probability that he will have tofu on Friday?
        \item Find the long term distribution for breakfast choices for Professor Hay.
    \end{enumerate}
\end{puzzle}



\begin{puzzle}
    Many Russians have experienced a sharp decline in their living standards due to President Yeltsin’s reforms. As a result, in the parliamentary elections held in December 1995, Communists and Nationalists made significant gains, and a new pattern in switching political parties emerged. The transition matrix for such a change is given below, where Communists, Nationalists, and Reformists are denoted by the letters C, N, and R, respectively.

    \[
        \begin{array}{lc|ccc}
                 &   &    & To &    \\
                 &   & C  & N  & R  \\
            \hline
                 & C & .5 & .4 & .1 \\
            From & N & .3 & .4 & .3 \\
                 & R & .2 & .2 & .6 \\
        \end{array}
    \]

    Find the following:
    \begin{enumerate}
        \item If in this election Communists received 25\% of the votes, Nationalists 30\%, and Reformists the rest 45\%, what will the distribution be in the next election?
        \item What will the distribution be in the third election?
        \item What will the distribution be in the fourth election?
        \item Determine the long term distribution.
    \end{enumerate}
\end{puzzle}
