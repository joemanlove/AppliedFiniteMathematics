\section{Simple Interest and Discount}

In this section, you will learn to:
\begin{enumerate}
    \item Find simple interest.
    \item Find present value.
    \item Find discounts and proceeds.
\end{enumerate}

\subsection{Simple Interest}
It costs to borrow money. The rent one pays for the use of money is called the interest. The amount of money that is being borrowed or loaned is called the principal or present value. Simple interest is paid only on the original amount borrowed. When the money is loaned out, the person who borrows the money generally pays a fixed rate of interest on the principal for the time period he keeps the money. Although the interest rate is often specified for a year, it may be specified for a week, a month, or a quarter, etc. The credit card companies often list their charges as monthly rates, sometimes it is as high as 1.5\% a month.

\begin{summarybox}{Simple Interest}

    If an amount \( P \) is borrowed for a time \( t \) at an interest rate of \( r \) per time period, then the simple interest is given by
    \[ I = P \cdot r \cdot t \]

    The total amount \( A \), also called the accumulated value or the future value, is given by
    \[ A = P + I = P + Prt \]
    or
    \[ A = P(1 + rt) \]
    where interest rate \( r \) is expressed in decimals.
\end{summarybox}

\begin{example}
    Ursula borrows \$600 for 5 months at a simple interest rate of 15\% per year. Find the interest, and the total amount she is obligated to pay?
\end{example}

\begin{solution}
    The interest is computed by multiplying the principal with the interest rate and the time.
    \[ I = Prt \]
    \[ I = \$600 (0.15) \frac{5}{12} = \$37.50 \]
    The total amount is
    \[ A = P + I = \$600 + \$37.50 = \$637.50 \]
    Incidentally, the total amount can be computed directly as
    \[ A = P(1 + rt) = \$600[1 + (0.15)(5/12)] = \$600(1 + 0.0625) = \$637.50 \]
\end{solution}

\begin{example}
    Jose deposited \$2500 in an account that pays 6\% simple interest. How much money will he have at the end of 3 years?
\end{example}

\begin{solution}
    The total amount or the future value is given by \( A = P(1 + rt) \).
    \[ A = \$2500[1 + (0.06)(3)] \]
    \[ A = \$2950 \]
\end{solution}

\begin{example}
    Darnel owes a total of \$3060 which includes 12\% interest for the three years he borrowed the money. How much did he originally borrow?
\end{example}
\begin{solution}
    This time we are asked to compute the principal \( P \).
    \[ \$3060 = P[1 + (.12)(3)] \]
    \[ \$3060 = P(1.36) \]
    \[ P = \frac{\$3060}{1.36} \]
    \[ P = \$2250 \]
    Darnel originally borrowed \$2250.
\end{solution}

\begin{example}
    A Visa credit card company charges a 1.5\% finance charge each month on the unpaid balance. If Martha owed \$2350 and has not paid her bill for three months, how much does she owe now?
\end{example}
\begin{solution}
    Before we attempt the problem, the reader should note that in this problem the rate of finance charge is given per month and not per year.

    The total amount Martha owes is the previous unpaid balance plus the finance charge.
    \[ A = \$2350 + \$2350(0.015)(3) = \$2350 + \$105.75 = \$2455.75 \]
    Alternatively, again, we can compute the amount directly by using formula \( A = P(1 + rt) \)
    \[ A = \$2350[1 + (.015)(3)] = \$2350(1.045) = \$2455.75 \]
\end{solution}

\subsection{Discounts and Proceeds}
Banks often deduct the simple interest from the loan amount at the time that the loan is made.  When this happens, we say the loan has been discounted.  The interest that is deducted is called the discount, and the actual amount that is given to the borrower is called the proceeds.  The amount the borrower is obligated to repay is called the maturity value.

\begin{summarybox}{Discounts and Proceeds}

    If an amount \( M \) is borrowed for a time \( t \) at a discount rate of \( r \) per year, then the discount \( D \) is
    \[ D = M \cdot r \cdot t \]

    The proceeds \( P \), the actual amount the borrower gets, is given by
    \[ P = M - D \]
    \[ P = M - Mrt \]
    or
    \[ P = M(1 - rt) \]
    where interest rate \( r \) is expressed in decimals.
\end{summarybox}

\begin{example}
    Francisco borrows \$1200 for 10 months at a simple interest rate of 15\% per year. Determine the discount and the proceeds.
\end{example}
\begin{solution}
    The discount \( D \) is the interest on the loan that the bank deducts from the loan amount.
    \[ D = Mrt \]
    \[ D = \$1200 \left(0.15\right) \left(\frac{10}{12}\right) = \$150 \]
    Therefore, the bank deducts \$150 from the maturity value of \$1200, and gives Francisco \$1050. Francisco is obligated to repay the bank \$1200.
    In this case, the discount \( D \) = \$150, and the proceeds \( P \) = \$1200 - \$150 = \$1050.
\end{solution}

\begin{example}
    If Francisco wants to receive \$1200 for 10 months at a simple interest rate of 15\% per year, what amount of loan should he apply for?
\end{example}
\begin{solution}
    In this problem, we are given the proceeds \( P \) and are being asked to find the maturity value \( M \).
    \[ P = \$1200, \quad r = 0.15, \quad t = \frac{10}{12} \]. We need to find \( M \).
    We know \( P = M - D \)
    but also \( D = Mrt \)
    therefore \( P = M - Mrt = M(1 - rt) \)
    \[ \$1200 = M\left[1 - (0.15)\left(\frac{10}{12}\right)\right] \]
    We need to solve for \( M \).
    \[ \$1200 = M(1 - 0.125) \]
    \[ \$1200 = M(0.875) \]
    \[ \frac{\$1200}{0.875} = M \]
    \[ \$1371.43 = M \]
    Therefore, Francisco should ask for a loan for \$1371.43.
    The bank will discount \$171.43 and Francisco will receive \$1200.
\end{solution}

