\section{Financial Calculations Vocabulary}

As we’ve seen in these examples, it’s important to read the problems carefully to correctly identify the situation.  It is essential to understand to vocabulary for financial problems.  Many of the vocabulary words used are listed in the glossary below for easy reference.

\begin{definition}
    The \textbf{Term}, denoted by $t$, is the time period for a loan or investment.  In this book $t$ is represented in years and should be converted into years when it is stated in months or other units.
\end{definition}

\begin{definition}
    The \textbf{Principal}, denoted by \(P\), refers to the amount of money borrowed in a loan. If a sum of money is invested for a period of time, the sum invested at the start is the Principal.
\end{definition}

\begin{definition}
    The \textbf{Present Value}, denoted by \(P\), is the value of money at the beginning of the time period.
\end{definition}

\begin{definition}
    The \textbf{Accumulated Value} or \textbf{Future Value} refers to the value of money at the end of the time period.
\end{definition}

\begin{definition}
    The \textbf{Discount} occurs in loans involving simple interest if the interest is deducted from the loan amount at the beginning of the loan period, rather than being repaid at the end of the loan period.
\end{definition}

\begin{definition}
    The \textbf{Periodic Payment}, denoted by \(m\), is the amount of a constant periodic payment that occurs at regular intervals during the time period under consideration, such as periodic payments made to repay a loan, regular periodic payments into a bank account as savings, or regular periodic payments to a retired person as an annuity.
\end{definition}

\begin{definition}
    The \textbf{Number of payment periods and compounding periods per year}, denoted by \(n\), is considered to be the same in this book when dealing with periodic payments. While in general the compounding and payment periods do not have to be the same and calculations can become more complicated, formulas for different periods can be found in finance textbooks or various online resources. Technology such as online financial calculators, spreadsheet financial functions, or financial pocket calculators can be utilized for these calculations.
\end{definition}

\begin{definition}
    The \textbf{Number of periods}, denoted by \(nt\), is calculated as \(nt = (\text{number of periods per year})(\text{number of years})\). It gives the total number of payment and compounding periods.
\end{definition}

\begin{definition}
    The \textbf{Annual interest rate} or \textbf{Nominal rate} is the stated annual interest rate. This is expressed as a percent but converted to decimal form when used in financial calculation formulas. For example, if a bank account pays 3\% interest compounded quarterly, then 3\% is the nominal rate, and it is included in the financial formulas as \(r = 0.03\).
\end{definition}

\begin{definition}
    The \textbf{Interest rate per compounding period}, \( \frac{r}{n} \), is the interest rate for each compounding period. If a bank account pays 3\% interest compounded quarterly, then \( \frac{r}{n} = \frac{0.03}{4} = 0.0075 \), corresponding to a rate of 0.75\% per quarter. Some Finite Math books use the symbol \(i\) to represent \( \frac{r}{n} \).
\end{definition}

\begin{definition}
    The \textbf{Effective Rate}, denoted by \( r_{effective} \), or \textbf{Effective Annual Interest Rate}, also known as \textbf{Annual Percentage Yield (APY)} or \textbf{Annual Percentage Rate (APR)}, is the interest rate compounded annually that would yield the same interest as the stated compounded rate for the investment. The effective rate provides a uniform way for investors or borrowers to compare different interest rates with different compounding periods.
\end{definition}

\begin{definition}
    \textbf{Interest}, denoted by \( I \), is money paid by a borrower for the use of money borrowed as a loan. It is also money earned over time when depositing money into a savings account, certificate of deposit, or money market account. When a person deposits money in a bank account, the depositor is essentially lending money to the bank temporarily, and the bank pays interest to the depositor.
\end{definition}

\begin{definition}
    A \textbf{Sinking Fund} is a fund established by making periodic payments into a savings or investment account over a period of time. The purpose of a sinking fund is to save for a future purchase, such as a business setting aside money to buy equipment at the end of the savings period.
\end{definition}

\begin{definition}
    An \textbf{Annuity} is a series of periodic payments. In this book, it refers to a stream of constant periodic payments made at the end of each compounding period for a certain amount of time. Commonly, the term annuity is used to describe a steady stream of payments received by an individual as retirement income, like from a pension. Annuity payments may be made at the end of each payment period (ordinary annuity) or at the beginning (annuity due). While compounding and payment periods can differ, this textbook only addresses cases where these periods are the same.
\end{definition}

\begin{definition}
    A \textbf{Lump Sum} refers to a single sum of money paid or deposited all at once, rather than distributed over time. An example includes lottery winnings when the recipient opts for a one-time "lump sum" payment instead of periodic payments over time. The term "lump sum" implies that the transaction is a one-off and not a sequence of periodic payments.
\end{definition}

\begin{definition}
    A \textbf{Loan} is an amount of money borrowed with an agreement that the borrower will repay the lender in the future, within a specified period known as the term of the loan. Repayment typically occurs through periodic payments until the loan is fully paid off by the end of the term. Some loans may be repaid in a single sum at the loan's end, with interest paid either periodically during the term or as a lump sum at the end, or through a discount at the start of the loan.
\end{definition}
