\section{Systems of Linear Equations - Special Cases}

\begin{puzzle}
    \begin{align*}
        2x + 6y & = 8 \\
        x + 3y  & = 4
    \end{align*}
\end{puzzle}

\begin{puzzle}
    The sum of the digits of a two-digit number is 9. The sum of the number and the number obtained by interchanging the digits is 99. Find the number.
\end{puzzle}

\begin{puzzle}
    \begin{align*}
        2x - y   & = 10 \\
        -4x + 2y & = 15
    \end{align*}
\end{puzzle}

\begin{puzzle}
    \begin{align*}
        x + y + z    & = 6  \\
        3x + 2y + z  & = 14 \\
        4x + 3y + 2z & = 20
    \end{align*}
\end{puzzle}

\begin{puzzle}
    \begin{align*}
        x + 2y - 4z  & = 1 \\
        2x - 3y + 8z & = 9
    \end{align*}
\end{puzzle}

\begin{puzzle}
    \begin{align*}
        x + 2y - 4z  & = 1 \\
        2x - 3y + 8z & = 9
    \end{align*}
\end{puzzle}

\begin{puzzle}
    Jessi has a collection of 15 coins consisting of nickels, dimes, and quarters. If the total worth of the coins is \$1.80, how many are there of each? Find all three solutions.
\end{puzzle}

\begin{puzzle}
    The latest reports indicate that there are altogether 20,000 American, French, and Russian troops in Bosnia. The sum of the number of Russian troops and twice the American troops equals 10,000. Furthermore, the Americans have 5,000 more troops than the French. Are these reports consistent?
\end{puzzle}

\begin{puzzle}
    \begin{align*}
        x + y + 2z   & = 0 \\
        x + 2y + z   & = 0 \\
        2x + 3y + 3z & = 0
    \end{align*}
\end{puzzle}

\begin{puzzle}
    Find three solutions to the following system of equations.
    \begin{align*}
        x + 2y + z & = 12 \\
        y          & = 3
    \end{align*}
\end{puzzle}

\begin{puzzle}
    For what values of \( k \) the following system of equations have a) No solution? b) Infinitely many solutions?
    \begin{align*}
        x + 2y  & = 5 \\
        2x + 4y & = k
    \end{align*}
\end{puzzle}

\begin{puzzle}
    \begin{align*}
        x + 3y - z & = 5
    \end{align*}
\end{puzzle}

\begin{puzzle}
    Why is it not possible for a linear system to have exactly two solutions? Explain geometrically.
\end{puzzle}