\section{Chapter Review}

\begin{puzzle}
    To reinforce her diet, Mrs. Tam bought a bottle containing 30 tablets of Supplement A and a bottle containing 50 tablets of Supplement B. Each tablet of supplement A contains 1000 mg of calcium, 400 mg of magnesium, and 15 mg of zinc, and each tablet of supplement B contains 800 mg of calcium, 500 mg of magnesium, and 20 mg of zinc.
    \begin{enumerate}
        \item Represent the amount of calcium, magnesium, and zinc in each tablet as a \(2 \times 3\) matrix.
        \item Represent the number of tablets in each bottle as a row matrix.
        \item Use matrix multiplication to determine the total amount of calcium, magnesium, and zinc in both bottles.
    \end{enumerate}
\end{puzzle}

\begin{puzzle}
    Let matrix \( A \) be
    $\begin{bmatrix}
            1 & -1 & 3 \\
            3 & -2 & 1
        \end{bmatrix}$
    and \( B \) be
    $\begin{bmatrix}
            3 & 3 & -1 \\
            1 & 4 & -3
        \end{bmatrix}$.
    Find the following:
    \begin{enumerate}
        \item \(\frac{1}{2}(A + B)\)
        \item \(3A - 2B\)
    \end{enumerate}
\end{puzzle}

\begin{puzzle}
    Let matrix \(C = \begin{bmatrix}
        1 & 1 & -1 \\
        2 & 1 & 1  \\
        1 & 0 & 1
    \end{bmatrix}\)
    and \(D = \begin{bmatrix}
        2 & -3 & -1 \\
        3 & -1 & -2 \\
        3 & -3 & -2
    \end{bmatrix}\). Find the following.
    \begin{enumerate}
        \item \(2(C - D)\)
        \item \(C - 3D\)
    \end{enumerate}
\end{puzzle}

\begin{puzzle}
    Let matrix \(E = \begin{bmatrix}
        1 & -1 \\
        2 & 3  \\
        1 & 2
    \end{bmatrix}\)
    and \(F = \begin{bmatrix}
        2 & 1 & -1 \\
        1 & 2 & -3
    \end{bmatrix}\). Find the following.
    \begin{enumerate}
        \item \(2EF\)
        \item \(3FE\)
    \end{enumerate}
\end{puzzle}

\begin{puzzle}
    Let matrix \(G = \begin{bmatrix}
        1 & -1 & 3 \\
        3 & 2  & 1
    \end{bmatrix}\)
    and \(H = \begin{bmatrix}
        a & b \\
        c & d \\
        e & f
    \end{bmatrix}\). Find the following.
    \begin{enumerate}
        \item \(2GH\)
        \item \(HG\)
    \end{enumerate}
\end{puzzle}

\begin{puzzle}
    Solve the following systems using the Gauss-Jordan Method.
    \begin{enumerate}
        \item
              \[\begin{aligned}
                      x + 3y - 2z  & = 7 \\
                      2x + 7y - 5z & = 1 \\
                      x + 5y - 3z  & = 1
                  \end{aligned}\]

        \item
              \[\begin{aligned}
                      2x - 4y + 4z & = 2 \\
                      2x + y + 9z  & = 1 \\
                      3x - 2y + 2z & = 7
                  \end{aligned}\]
    \end{enumerate}
\end{puzzle}

\begin{puzzle}
    An apple, a banana and three oranges or two apples, two bananas, and an orange, or four bananas
    and two oranges cost \$2. Find the price of each.
\end{puzzle}

\begin{puzzle}
    Solve the following systems. If a system has an infinite number of solutions, first express the solution in parametric form, and then determine one particular solution.
    \begin{enumerate}
        \item
              \[\begin{aligned}
                      x + y + z    & = 6 \\
                      2x - 3y + 2z & = 1 \\
                      3x - 2y + 3z & = 1 \\
                  \end{aligned}\]

        \item
              \[\begin{aligned}
                      x + y + 3z & = 4 \\
                      x + z      & = 1 \\
                      2x - y     & = 2
                  \end{aligned}\]
    \end{enumerate}
\end{puzzle}

\begin{puzzle}
    Else has a collection of 12 coins consisting of nickels, dimes and quarters. If the total worth of the coins is \$1.80, how many are there of each? Find all possible solutions.
\end{puzzle}

\begin{puzzle}{Solve the following systems}
    If a system has an infinite number of solutions, first express the solution in parametric form, and then find a particular solution.
    \begin{enumerate}
        \item
              \[\begin{aligned}
                      x + 2y       & = 4 \\
                      2x + 4y      & = 8 \\
                      3x + 6y - 3z & = 3
                  \end{aligned}\]

        \item
              \[\begin{aligned}
                      x - 2y + 2z  & = 1 \\
                      2x - 3y + 5z & = 4
                  \end{aligned}\]
    \end{enumerate}
\end{puzzle}

\begin{puzzle}{Solve the following systems}
    If a system has an infinite number of solutions, first express the solution in parametric form, and then provide one particular solution.
    \begin{enumerate}
        \item
              \[\begin{aligned}
                      2x + y - 2z  & = 0 \\
                      2x + 2y - 3z & = 0 \\
                      6x + 4y - 7z & = 0 \\
                  \end{aligned}\]

        \item
              \[\begin{aligned}
                      3x + 4y - 3z & = 5 \\
                      2x + 3y - z  & = 4 \\
                      x + 2y + z   & = 1
                  \end{aligned}\]
    \end{enumerate}
\end{puzzle}

\begin{puzzle} Find the inverse of the following matrices.
    \begin{enumerate}
        \item
              \[\begin{bmatrix}
                      2 & 3 \\
                      3 & 5
                  \end{bmatrix}\]

        \item
              \[\begin{bmatrix}
                      1 & 1 & 1 \\
                      1 & 2 & 1 \\
                      2 & 3 & 1
                  \end{bmatrix}\]
    \end{enumerate}
\end{puzzle}

\begin{puzzle} Solve the following systems using the matrix inverse method.
    \begin{enumerate}
        \item
              \[\begin{aligned}
                      2x + 3y + z & = 1 \\
                      x  + 2y + z & = 9 \\
                      x  +  y + z & = 5 \\
                  \end{aligned}\]

        \item
              \[\begin{aligned}
                      x  + 2y - 3z + w & = 7  \\
                      x  - z           & = 4  \\
                      x  - 2y + z      & = 0  \\
                      y  - 2z + w      & = -1
                  \end{aligned}\]
    \end{enumerate}
\end{puzzle}

\begin{puzzle}{Use matrix A, given below, to encode the following messages. The space between the letters is represented by the number 27, and all punctuation is ignored.}
    \[ A = \begin{pmatrix}
            1 & 2 & 0 \\
            1 & 2 & 1 \\
            0 & 1 & 0
        \end{pmatrix} \]
    \begin{enumerate}
        \item TAKE IT AND RUN.
        \item GET OUT QUICK.
    \end{enumerate}
\end{puzzle}

\begin{puzzle}{Decode the following messages that were encoded using matrix A in the above problem.}
    \begin{enumerate}
        \item 44, 71, 15, 18, 27, 1, 68, 82, 27, 69, 76, 27, 19, 33, 9
        \item 37, 64, 15, 36, 54, 15, 67, 75, 20, 59, 66, 27, 39, 43, 12
    \end{enumerate}
\end{puzzle}


\begin{puzzle}
    Chris, Bob, and Matt decide to help each other study during the final exams. Chris's favorite
    subject is chemistry, Bob loves biology, and Matt knows his math. Each studies his own subject
    as well as helps the others learn their subjects. After the finals, they realize that Chris spent 40\%
    of his time studying his own subject chemistry, 30\% of his time helping Bob learn chemistry, and
    30\% of the time helping Matt learn chemistry. Bob spent 30\% of his time studying his own subject
    biology, 30\% of his time helping Chris learn biology, and 40\% of the time helping Matt learn biology.
    Matt spent 20\% of his time studying his own subject math, 40\% of his time helping Chris learn math, and 40\% of the time helping Bob learn math. If they originally agreed that each should work
    about 33 hours, how long did each work?
\end{puzzle}

\begin{puzzle}
    As in the previous problem, Chris, Bob, and Matt decide to not only help each other study during
    the final exams, but also tutor others to make a little money. Chris spends 30\% of his time studying
    chemistry, 15\% of his time helping Bob with chemistry, and 25\% helping Matt with chemistry. Bob
    spends 25\% of his time studying biology, 15\% helping Chris with biology, and 30\% helping Matt.
    Similarly, Matt spends 20\% of his time on his own math, 20\% helping Chris, and 20\% helping Bob.
    If they spend respectively, 12, 12, and 10 hours tutoring others, how many total hours are they
    going to end up working?
\end{puzzle}