\section{Sample Spaces and Probability Puzzles}

Write a sample space for the given experiment.

\begin{puzzle}
    A die is rolled.
\end{puzzle}

\begin{puzzle}
    A penny and a nickel are tossed.
\end{puzzle}

\begin{puzzle}
    A die is rolled, and a coin is tossed.
\end{puzzle}

\begin{puzzle}
    Three coins are tossed.
\end{puzzle}

\begin{puzzle}
    Two dice are rolled.
\end{puzzle}

\begin{puzzle}
    A jar contains four marbles numbered 1, 2, 3, and 4. Two marbles are drawn.
\end{puzzle}


A card is selected from a deck. Find the following probabilities

\begin{puzzle}
    P(\text{an ace})
\end{puzzle}

\begin{puzzle}
    P(\text{a red card})
\end{puzzle}

\begin{puzzle}
    P(\text{a club})
\end{puzzle}

\begin{puzzle}
    P(\text{a face card})
\end{puzzle}

\begin{puzzle}
    P (\text{a jack or spade})
\end{puzzle}

\begin{puzzle}
    P (\text{a jack and a spade})
\end{puzzle}

A jar contains 6 red, 7 white, and 7 blue marbles. If a marble is chosen at random, find the following
probabilities.

\begin{puzzle}
    P(\text{red})
\end{puzzle}

\begin{puzzle}
    P(\text{white})
\end{puzzle}

\begin{puzzle}
    P(\text{red or blue})
\end{puzzle}

\begin{puzzle}
    P(\text{red and blue})
\end{puzzle}




Three coins are tossed, find the following probabilities:

\begin{puzzle}
    P(\text{two heads and a tails})
\end{puzzle}

\begin{puzzle}
    P(\text{at least one head})
\end{puzzle}

\begin{puzzle}
    P(\text{at least one of each})
\end{puzzle}

\begin{puzzle}
    P(\text{at most one tails})
\end{puzzle}

Two dice are rolled. Find the following probabilities.

\begin{puzzle}
    P(\text{the sum of the dice is 5})
\end{puzzle}

\begin{puzzle}
    P(\text{the sum of the dice is 8})
\end{puzzle}

\begin{puzzle}
    P(\text{the sum is 3 or 6})
\end{puzzle}

\begin{puzzle}
    P(\text{the sum is more than 10})
\end{puzzle}

A jar contains four marbles numbered 1, 2, 3, and 4. If two marbles are drawn, find the following probabilities

\begin{puzzle}
    P(\text{the sum of the number is 5})
\end{puzzle}

\begin{puzzle}
    P(\text{the sum of the numbers is odd})
\end{puzzle}

\begin{puzzle}
    P(\text{the sum of the numbers is 9})
\end{puzzle}

\begin{puzzle}
    P(\text{one of the numbers is 3})
\end{puzzle}
