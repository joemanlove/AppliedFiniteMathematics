\section{Tree Diagrams and Combination Probability Puzzles}


Two apples are chosen from a basket containing five red and three yellow apples (without replacement). Draw a tree diagram below,
and find the following probabilities

\begin{puzzle}
    \( P(\text{both red}) \)
\end{puzzle}

\begin{puzzle}
    \( P(\text{one red, one yellow}) \)
\end{puzzle}

\begin{puzzle}
    \( P(\text{both yellow}) \)
\end{puzzle}

\begin{puzzle}
    \( P(\text{first red and second yellow}) \)
\end{puzzle}

A basket contains six red and four blue marbles. Three marbles are drawn at random (without replacement). Find the following
probabilities without using combinations.

\begin{puzzle}
    \( P(\text{All three red})\)
\end{puzzle}

\begin{puzzle}
    \( P(\text{(two red, one blue)})\)
\end{puzzle}

\begin{puzzle}
    \( P(\text{one red, two blue})\)
\end{puzzle}

\begin{puzzle}
    \( P(\text{first red, second blue, third red})\)
\end{puzzle}

Three marbles are drawn from a jar containing five red, four white, and three blue marbles (without replacement). Find the
following probabilities using combinations.


\begin{puzzle}
    \( P(\text{all three red})\)
\end{puzzle}

\begin{puzzle}
    \( P(\text{(two white and 1 blue})\)
\end{puzzle}

\begin{puzzle}
    \( P(\text{none white})\)
\end{puzzle}

\begin{puzzle}
    \( P(\text{at least one red})\)
\end{puzzle}


A committee of four is selected from a total of 4 freshmen, 5 sophomores, and 6 juniors. Find the probabilities
for the following events.
\begin{puzzle}
    At least three freshmen.
\end{puzzle}

\begin{puzzle}
    No sophomores.
\end{puzzle}

\begin{puzzle}
    All four of the same class.
\end{puzzle}

\begin{puzzle}
    Not all four from the same class.
\end{puzzle}

\begin{puzzle}
    Exactly three of the same class.
\end{puzzle}

\begin{puzzle}
    More juniors than freshmen and sophomores combined.
\end{puzzle}


Five cards are drawn from a deck. Find the probabilities for the following events

\begin{puzzle}
    Two hearts, two spades, and one club.
\end{puzzle}

\begin{puzzle}
    A flush of any suit (all cards of a single suit).
\end{puzzle}

\begin{puzzle}
    A full house of nines and tens (3 nines and 2 tens).
\end{puzzle}

\begin{puzzle}
    Any full house.
\end{puzzle}

\begin{puzzle}
    A pair of nines and tens.
\end{puzzle}

\begin{puzzle}
    Two pairs.
\end{puzzle}

Do the following birthday problems. You may ignore leap years.

\begin{puzzle}
    If there are five people in a room, what is the probability that no two have the same birthday?

\end{puzzle}

\begin{puzzle}
    If there are five people in a room, what is the probability that at least two people have the same
    birthday?
\end{puzzle}