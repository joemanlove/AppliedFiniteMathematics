
\section{Non-Strictly Determined Games}\label{section_nonstrictly_determined_games}

In this section, you will learn to:
\begin{enumerate}
    \item find the optimal strategy for non-strictly determined games.
    \item find the value of non-strictly determined games.
\end{enumerate}

In this section, we study games that have no saddle points.  This means that these games do not possess a pure strategy.  We call these games non-strictly determined games.  If the game is played only once, it will make no difference what move is made.   However, if the game is played repeatedly, a mixed strategy consisting of alternating random moves can be worked out.

We consider the following example.

\begin{example}\label{example_matching_coins}
    Suppose Robert and Carol decide to play a game using a dime and a quarter. At a given signal, they simultaneously show one of the two coins. If the coins match, Robert gets both coins, but if they don't match, Carol gets both coins. Determine whether the game is strictly determined.
\end{example}

\begin{solution}
    We write the payoff table for Robert as follows:

    \[
        \begin{array}{cc|cc}
                          &                & \text{Carol} &                \\
                          &                & \text{Dime}  & \text{Quarter} \\
            \hline
            \text{Robert} & \text{Dime}    & 10           & -10            \\
                          & \text{Quarter} & -25          & 25             \\
        \end{array}
    \]

    To determine whether the game is strictly determined, we look for a saddle point. Again, we place an asterisk next to the minimum entry in each row, and a box around the maximum value in each column. We get

    \[
        \begin{array}{cc|cc}
                          &                & \text{Carol} &                \\
                          &                & \text{Dime}  & \text{Quarter} \\
            \hline
            \text{Robert} & \text{Dime}    & \fbox{10}    & -10\text{*}    \\
                          & \text{Quarter} & -25\text{*}  & \fbox{25}      \\
        \end{array}
    \]

    Since there is no entry that has both an asterisk and a box, the game does not have a saddle point, and thus it is non-strictly determined.
\end{solution}

We wish to devise a strategy for Robert.   If Robert consistently shows a dime, for example, Carol will see the pattern and will start showing a quarter, and Robert will lose.  Conversely, if Carol repeatedly shows a quarter, Robert will start showing a quarter, thus resulting in Carol's loss.  So a good strategy is to throw your opponent off by showing a dime some of the times and showing a quarter other times.  Before we develop an optimal strategy for each player, we will consider an arbitrary strategy for each and determine the corresponding payoffs.

\begin{example}
    Suppose in Example \ref{example_matching_coins}, Robert decides to show a dime with .20 probability and a quarter with .80 probability, and Carol decides to show a dime with .70 probability and a quarter with .30 probability. What is the expected payoff for Robert?
\end{example}

\begin{solution}
    Let \( R \) denote Robert's strategy and \( C \) denote Carol's strategy. Since Robert is a row player and Carol is a column player, their strategies are written as follows:
    \[ R = \begin{bmatrix} .20 & .80 \end{bmatrix}, \quad C = \begin{bmatrix} .70 \\ .30 \end{bmatrix} \]

    To find the expected payoff, we use the following reasoning.
    Since Robert chooses to play row 1 with .20 probability and Carol chooses to play column 1 with .70 probability, the move row 1, column 1 will be chosen with \((.20)(.70) = .14\) probability. The fact that this move has a payoff of 10 cents for Robert, Robert's expected payoff for this move is \((.14)(.10) = .014\) cents. Similarly, we compute Robert's expected payoffs for the other cases. The table below lists expected payoffs for all four cases.

    \begin{center}
        \begin{tabular}{|l|c|c|c|}
            \hline
            Move            & Probability          & Payoff        & Expected Payoff \\
            \hline
            Row 1, Column 1 & \((.20)(.70) = .14\) & 10 cents      & 1.4 cents       \\
            Row 1, Column 2 & \((.20)(.30) = .06\) & \(-10\) cents & \(-.6\) cents   \\
            Row 2, Column 1 & \((.80)(.70) = .56\) & \(-25\) cents & \(-14\) cents   \\
            Row 2, Column 2 & \((.80)(.30) = .24\) & 25 cents      & 6.0 cents       \\
            \hline
            Totals          & 1                    &               & \(-7.2\) cents  \\
            \hline
        \end{tabular}
    \end{center}

    The above table shows that if Robert plays the game with the strategy \( R = \begin{bmatrix} .20 & .80 \end{bmatrix} \) and Carol plays with the strategy \( C = \begin{bmatrix} .70 \\ .30 \end{bmatrix} \), Robert can expect to lose 7.2 cents for every game.

    Alternatively, if we call the game matrix \( G \), then the expected payoff for the row player can be determined by multiplying matrices \( R \), \( G \) and \( C \). Thus, the expected payoff \( P \) for Robert is as follows:

    \[
        P = RGC = \begin{bmatrix} .20 & .80 \end{bmatrix} \begin{bmatrix} 10 & -10 \\ -25 & 25 \end{bmatrix} \begin{bmatrix} .70 \\ .30 \end{bmatrix} = -7.2 \text{ cents}.
    \]

    which is the same as the one obtained from the table.

\end{solution}

\begin{example}
    For the following game matrix \( G \), determine the optimal strategy for both the row player and the column player, and find the value of the game.
    \[ G = \begin{bmatrix} 1 & -2 \\ -3 & 4 \end{bmatrix} \]
\end{example}
\begin{solution}
    Let us suppose that the row player uses the strategy \( R = \begin{bmatrix} r & 1 - r \end{bmatrix} \). Now if the column player plays column 1, the expected payoff \( P \) for the row player is:
    \[ P(r) = 1(r) + (-3)(1 - r) = 4r - 3. \]
    This can also be computed as follows:
    \[ P(r) = \begin{bmatrix} r & 1 - r \end{bmatrix} \begin{bmatrix} 1 \\ -3 \end{bmatrix} = 4r - 3. \]



    If the row player plays the strategy \( \begin{bmatrix} r & 1-r \end{bmatrix} \) and the column player plays column 2, the expected payoff \( P \) for the row player is
    \[
        P(r) = \begin{bmatrix} r & 1-r \end{bmatrix} \begin{bmatrix} -2 \\ 4 \end{bmatrix} = -6r + 4.
    \]
    We have two equations: \( P(r) = 4r - 3 \) and \( P(r) = -6r + 4 \)

    The row player is trying to improve upon his worst scenario, and that only happens when the two lines intersect. Any point other than the point of intersection will not result in optimal strategy as one of the expectations will fall short.

    Solving for \( r \) algebraically, we get
    \[
        4r - 3 = -6r + 4
    \]
    \[
        r = \frac{7}{10}.
    \]
    Therefore, the optimal strategy for the row player is \( \begin{bmatrix} .7 & .3 \end{bmatrix} \).

    Alternatively, we can find the optimal strategy for the row player by, first, multiplying the row matrix with the game matrix as shown below.
    \[
        \begin{bmatrix} r & 1-r \end{bmatrix} \begin{bmatrix} 1 & -2 \\ -3 & 4 \end{bmatrix} = \begin{bmatrix} 4r - 3 & -6r + 4 \end{bmatrix}
    \]
    And then by equating the two entries in the product matrix. Again, we get \( r = .7 \), which gives us the optimal strategy \( \begin{bmatrix} .7 & .3 \end{bmatrix} \).

    We use the same technique to find the optimal strategy for the column player.

    Suppose the column player's optimal strategy is represented by \( \begin{bmatrix} c & 1-c \end{bmatrix} \). We first multiply the game matrix by the column matrix as shown below.
    \[
        \begin{bmatrix}
            1  & -2 \\
            -3 & 4
        \end{bmatrix}
        \begin{bmatrix}
            c \\
            1-c
        \end{bmatrix}
        =
        \begin{bmatrix}
            3c-2 \\
            -7c+4
        \end{bmatrix}
    \]
    And then equate the entries in the product matrix. We get
    \[
        3c - 2 = -7c + 4
    \]
    \[
        c = .6
    \]
    Therefore, the column player's optimal strategy is \( \begin{bmatrix} .6 & .4 \end{bmatrix} \).

    To find the expected value, \( V \), of the game, we find the product of the matrices \( R \), \( G \), and \( C \).
    \[
        V =
        \begin{bmatrix} .7 & .3 \end{bmatrix}
        \begin{bmatrix}
            1  & -2 \\
            -3 & 4
        \end{bmatrix}
        \begin{bmatrix}
            .6 \\
            .4
        \end{bmatrix}
        = V = -0.2
    \]
    That is, if both players play their optimal strategies, the row player can expect to lose .2 units for every game.

\end{solution}

\begin{example}
    For the game in Example \ref{example_matching_coins}, determine the optimal strategy for both Robert and Carol, and find the value of the game. We rewrite the game matrix.
    \[ G = \begin{bmatrix} 10 & -10 \\ -25 & 25 \end{bmatrix} \]
\end{example}
\begin{solution}
    Let \( R = \begin{bmatrix} r & 1 - r \end{bmatrix} \) be Robert's strategy, and \( C = \begin{bmatrix} c \\ 1 - c \end{bmatrix} \) be Carol's strategy.

    To find the optimal strategy for Robert, we first find the product \( RG \) as below.
    \[ \begin{bmatrix} r & 1 - r \end{bmatrix} \begin{bmatrix} 10 & -10 \\ -25 & 25 \end{bmatrix} = \begin{bmatrix} 35r - 25 & -35r + 25 \end{bmatrix} \]
    By setting the entries equal, we get \[ 35r - 25 = -35r + 25 \] resulting in \( r = \frac{5}{7} \). Therefore, the optimal strategy for Robert is \[ R=  \begin{bmatrix} 5/7 & 2/7 \end{bmatrix} .\]

    To find the optimal strategy for Carol, we first find the following product.
    \[ \begin{bmatrix} 10 & -10 \\ -25 & 25 \end{bmatrix} \begin{bmatrix} c \\ 1 - c \end{bmatrix} = \begin{bmatrix} 20c - 10 \\ -50c +25 \end{bmatrix} \]
    We now set the entries equal to each other, and we get \[ 20c - 10 = -50c + 25 \] resulting in \( c = \frac{1}{2} \).

    Therefore, the optimal strategy for Carol is \[ C = \begin{bmatrix} 1/2 & 1/2 \end{bmatrix}. \]

    To find the expected value, \( V \), of the game, we find the product \( RGC \).

    \[
        V = \begin{bmatrix} 5/7 & 2/7 \end{bmatrix}
        \begin{bmatrix}
            10  & -10 \\
            -25 & 25
        \end{bmatrix}
        \begin{bmatrix} 1/2 \\ 1/2 \end{bmatrix}
        = \begin{bmatrix} 0 \end{bmatrix}
    \]

    If both players play their optimal strategy, the value of the game is zero. In such case, the game is called \textbf{fair}.

\end{solution}
