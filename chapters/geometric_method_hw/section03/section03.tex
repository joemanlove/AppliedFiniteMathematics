\section{Chapter Review}

\begin{puzzle}
    Mr. Shoemacher has \$20,000 to invest in two types of mutual funds, Coleman High-yield Fund,
    and Coleman Equity Fund. The High-yield fund gives an annual yield of 12\%, while the Equity
    fund earns 8\%. Mr. Shoemacher would like to invest at least \$3000 in the High-yield fund and at
    least \$4000 in the Equity fund. How much money should he invest in each to maximize his annual
    yield, and what is the maximum yield?

\end{puzzle}

\begin{puzzle}
    Dr. Lum teaches part-time at two different community colleges, Hilltop College and Serra College.
    Dr. Lum can teach up to 5 classes per semester. For every class taught by him at Hilltop College,
    he needs to spend 3 hours per week preparing lessons and grading papers, and for each class at
    Serra College, he must do 4 hours of work per week. He has determined that he cannot spend more
    than 18 hours per week preparing lessons and grading papers. If he earns \$4,000 per class at Hilltop
    College and \$5,000 per class at Serra College, how many classes should he teach at each college to
    maximize his income, and what will be his income?
\end{puzzle}

\begin{puzzle}
    Mr. Shamir employs two part-time typists, Inna and Jim for his typing needs. Inna charges \$10
    an hour and can type 6 pages an hour, while Jim charges \$12 an hour and can type 8 pages per
    hour. Each typist must be employed at least 8 hours per week to keep them on the payroll. If
    Mr. Shamir has at least 208 pages to be typed, how many hours per week should he employ each
    student to minimize his typing costs, and what will be the total cost?

\end{puzzle}

\begin{puzzle}
    Mr. Boutros wants to invest up to \$20,000 in two stocks, Cal Computers and Texas Tools. The Cal
    Computers stock is expected to yield a 16\% annual return, while the Texas Tools stock promises
    a 12\% yield. Mr. Boutros would like to earn at least \$2,880 this year. According to Value Line
    Magazine's safety index (1 highest to 5 lowest), Cal Computers has a safety number of 3 and Texas
    Tools has a safety number of 2. How much money should he invest in each to minimize the safety
    number? Note: A lower safety number means less risk.
\end{puzzle}

\begin{puzzle}
    A department store sells two types of televisions: Regular and Big Screen. The store can sell up
    to 90 sets a month. A Regular television requires 6 cubic feet of storage space, and a Big Screen
    television requires 18 cubic feet of space, and a maximum of 1080 cubic feet of storage space is
    available. The Regular and Big Screen televisions take up, respectively, 2 and 3 sales hours of
    labor, and a maximum of 198 hours of labor is available. If the profit made from each of these
    types is \$60 and \$80, respectively, how many of each type of television should be sold to maximize
    profit, and what is the maximum profit?

\end{puzzle}

\begin{puzzle}
    A company manufactures two types of printers, the Inkjet and the Laser. The Inkjet generates a
    profit of \$100 per printer and the Laser a profit of \$150. On the assembly line the Inkjet requires 7
    hours, while the Laser takes 11 hours. Both printers require one hour for testing. The Inkjet requires
    one hour and the Laser needs 3 hours for finishing. On a particular production run the company
    has available 1,540 work hours on the assembly line, 200 work hours in the testing department, and
    360 work hours for finishing. How many sets of each type should the company produce to maximize
    profit, and what is that maximum profit?
\end{puzzle}

\begin{puzzle}
    John wishes to choose a combination of two types of cereals for breakfast - Cereal A and Cereal
    B. A small box(one serving) of Cereal A costs \$0.50 and contains 10 units of vitamins, 5 units of
    minerals, and 15 calories. A small box(one serving) of Cereal B costs \$0.40 and contains 5 units
    of vitamins, 10 units of minerals, and 15 calories. John wants to buy enough boxes to have at
    least 500 units of vitamins, 600 units of minerals, and 1200 calories. How many boxes of each food
    should he buy to minimize his cost, and what is the minimum cost?
\end{puzzle}

\begin{puzzle}
    Jessica needs at least 60 units of vitamin A, 40 units of vitamin B, and 140 units of vitamin C
    each week. She can choose between Costless brand or Savemore brand tablets. A Costless tablet
    costs 5 cents and contains 3 units of vitamin A, 1 unit of vitamin B, and 2 units of vitamin C, and
    a Savemore tablet costs 7 cents and contains 1 unit of A, 1 of B, and 5 of C. How many tablets of
    each kind should she buy to minimize cost, and what is the minimum cost?
\end{puzzle}

\begin{puzzle}
    A small company manufactures two types of radios- regular and short-wave. The manufacturing of
    each radio requires three operations: Assembly, Finishing and Testing. The regular radios require
    1 hour of Assembly, 3 hours of Finishing, and 1 hour of Testing. The short-wave radios require 3
    hours of Assembly, 1 hour of Finishing, and 1 hour of Testing. The total work-hours available per
    week in the Assembly division is 60, in the Finishing division is 60, and in the Testing is 24. If a
    profit of \$50 is realized for every regular radio, and \$75 for every short-wave radio, how many of
    each should be manufactured to maximize profit, and what is the maximum profit?
\end{puzzle}

\begin{puzzle}
    A factory manufactures two products, A and B. Each product requires the use of three machines, Machine I, Machine II, and Machine III. The time requirements and total hours available on each machine are listed below.
    \begin{center}
        \begin{tabular}{|l|c|c|c|}
            \hline
                        & Machine I & Machine II & Machine III \\
            \hline
            Product A   & 1         & 2          & 4           \\
            \hline
            Product B   & 2         & 2          & 2           \\
            \hline
            Total hours & 70        & 90         & 160         \\
            \hline
        \end{tabular}
    \end{center}
    If product A generates a profit of \$60 per unit and product B a profit of \$50 per unit, how many units of each product should be manufactured to maximize profit, and what is the maximum profit?
\end{puzzle}

\begin{puzzle}
    A company produces three types of shoes, formal, casual, and athletic, at its two factories, Factory I and Factory II. Daily production of each factory for each type of shoe is listed below.
    \begin{center}
        \begin{tabular}{|l|c|c|}
            \hline
                     & Factory I & Factory II \\
            \hline
            Formal   & 100       & 100        \\
            \hline
            Casual   & 100       & 200        \\
            \hline
            Athletic & 300       & 100        \\
            \hline
        \end{tabular}

    \end{center}
    The company must produce at least 6000 pairs of formal shoes, 8000 pairs of casual shoes, and 9000 pairs of athletic shoes. If the cost of operating Factory I is \$1500 per day and the cost of operating Factory II is \$2000, how many days should each factory operate to produce the order at a minimum cost, and what is the minimum cost?
\end{puzzle}

\begin{puzzle}
    A professor gives two types of quizzes, objective and recall. He is planning to give at least 15
    quizzes this quarter. The student preparation time for an objective quiz is 15 minutes and for a
    recall quiz 30 minutes. The professor would like a student to spend at least 5 hours (300 minutes)
    preparing for these quizzes above and beyond the normal study time. The average score on an
    objective quiz is 7, and on a recall type 5, and the professor would like the students to score at
    least 85 points on all quizzes. It takes the professor one minute to grade an objective quiz, and 1.5
    minutes to grade a recall type quiz. How many of each type should he give in order to minimize
    his grading time?
\end{puzzle}

\begin{puzzle}
    A company makes two mixtures of nuts: Mixture A and Mixture B. Mixture A contains 30\%
    peanuts, 30\% almonds and 40\% cashews and sells for \$5 per pound. Mixture B contains 30\%
    peanuts, 60\% almonds and 10\% cashews and sells for \$3 a pound. The company has 540 pounds
    of peanuts, 900 pounds of almonds, 480 pounds of cashews. How many pounds of each of mixtures
    A and B should the company make to maximize profit, and what is the maximum profit?
\end{puzzle}
