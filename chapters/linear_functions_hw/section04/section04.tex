\section{Applications}

In the following application problems, assume a linear relationship holds.

\begin{puzzle}
    The variable cost to manufacture a product is \$25, and the fixed costs are \$1200. If \( x \) represents the number of items manufactured and \( y \) the cost, write the cost function.
\end{puzzle}

\begin{puzzle}
    It costs \$90 to rent a car driven 100 miles and \$140 for one driven 200 miles. If \( x \) is the number of miles driven and \( y \) the total cost of the rental, write the cost function.
\end{puzzle}

\begin{puzzle}
    The variable cost to manufacture an item is \$20, and it costs a total of \$750 to produce 20 items. If \( x \) represents the number of items manufactured and \( y \) the cost, write the cost function.
\end{puzzle}

\begin{puzzle}
    To manufacture 30 items, it costs \$2700, and to manufacture 50 items, it costs \$3200. If \( x \) represents the number of items manufactured and \( y \) the cost, write the cost function.
\end{puzzle}

\begin{puzzle}
    To manufacture 100 items, it costs \$32,000, and to manufacture 200 items, it costs \$40,000. If \( x \) represents the number of items manufactured and \( y \) the cost, write the cost function.
\end{puzzle}

\begin{puzzle}
    It costs \$1900 to manufacture 60 items, and the fixed costs are \$700. If \( x \) represents the number of items manufactured and \( y \) the cost, write the cost function.
\end{puzzle}

\begin{puzzle}
    A person who weighs 150 pounds has 60 pounds of muscles, and a person that weighs 180 pounds has 72 pounds of muscles. If \( x \) represents the body weight and \( y \) the muscle weight, write an equation describing their relationship. Use this relationship to determine the muscle weight of a person that weighs 170 pounds.
\end{puzzle}

\begin{puzzle}
    \item A spring on a door stretches 6 inches if a force of 30 pounds is applied, and it stretches 10 inches if a force of 50 pounds is applied. If \( x \) represents the number of inches stretched, and \( y \) the force applied, write an equation describing the relationship. Use this relationship to determine the amount of force required to stretch the spring 12 inches.
\end{puzzle}

\begin{puzzle}
    \item A male college student who is 64 inches tall weighs 110 pounds, and another student who is 74 inches tall weighs 180 pounds. Assuming the relationship between male students' heights (\( x \)), and weights (\( y \)) is linear, write a function to express weights in terms of heights, and use this function to predict the weight of a student who is 68 inches tall.
\end{puzzle}

\begin{puzzle}
    \item EZ Clean company has determined that if it spends \$30,000 on advertisement, it can hope to sell 12,000 of its Minivacs a year, but if it spends \$50,000, it can sell 16,000. Write an equation that gives a relationship between the number of dollars spent on advertisement (\( x \)) and the number of minivacs sold (\( y \)).
\end{puzzle}

\begin{puzzle}
    \item The freezing temperatures for Celsius and Fahrenheit scales are 0 degree and 32 degrees, respectively. The boiling temperatures for Celsius and Fahrenheit are 100 degrees and 212 degrees, respectively. Let \( C \) denote the temperature in Celsius and \( F \) in Fahrenheit. Write the conversion function from Celsius to Fahrenheit, and use this function to convert 25 degrees Celsius into an equivalent Fahrenheit measure.
\end{puzzle}

\begin{puzzle}
    \item By reversing the coordinates in the previous problem, find a conversion function that converts Fahrenheit into Celsius, and use this conversion function to convert 72 degrees Fahrenheit into an equivalent Celsius measure.
\end{puzzle}


% TODO Update numbers here
\begin{puzzle}
    \item The population of California in the year 1960 was 17 million, and in 1995 it was 32 million. Write the population function, and use this function to find the population of California in the year 2010. (Hint: Use the year 1960 as the base year, that is, assume 1960 as the year zero. This will make 1995, and 2010 as the years 35, and 50, respectively.)
\end{puzzle}

\begin{puzzle}
    \item In the U.S. the number of people infected with the HIV virus in 1985 was 1,000, and in 1995 that number became 350,000. If the increase in the number is linear, write an equation that will give the number of people infected in any year. If this trend continues, what will the number be in 2010? (Hint: See previous problem.)
\end{puzzle}


\begin{puzzle}
    \item In 1975, an average house in San Jose cost \$45,000 and the same house in 1995 costs \$195,000. Write an equation that will give the price of a house in any year, and use this equation to predict the price of a similar house in the year 2010.
\end{puzzle}

\begin{puzzle}
    \item An average math textbook cost \$25 in 1980, and \$60 in 1995. Write an equation that will give the price of a math book in any given year, and use this equation to predict the price of the book in 2010.
\end{puzzle}
