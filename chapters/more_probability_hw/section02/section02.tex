\section{Bayes Formula Puzzles}

Use both tree diagrams and Bayes' formula to solve the following problems.

\begin{puzzle}
    A telemarketing executive has determined that for a particular product, 20\% of the people contacted will purchase the product. If 10 people are contacted, what is the probability that at most 2 will buy the product?
\end{puzzle}

%TODO fix this problem?
\begin{puzzle}
    To the problem: "Five cards are dealt from a deck of cards, find the probability that three of them are kings," the following incorrect answer was offered by a student: \[ 5C3 \left(\frac{1}{13}\right)^3 \left(\frac{12}{13}\right)^2 \]
    What change would you make in the wording of the problem for the given answer to be correct?
\end{puzzle}

\begin{puzzle}
    Jar I contains five red and three white marbles, and Jar II contains four red and two white marbles. A jar is picked at random and a marble is drawn. Find the following probabilities:
    \begin{enumerate}
        \item \( P(\text{marble is red}) \)
        \item \( P(\text{It came from Jar II} | \text{the marble drawn is white}) \)
        \item \( P(\text{Red} | \text{Jar I}) \)
    \end{enumerate}
\end{puzzle}

\begin{puzzle}
    In Mr. Symons' class, if a person does their homework most days, their chance of passing the course is 90\%. Conversely, if a person does not do their homework, the chance of passing is only 20\%. It's claimed that 80\% of the students do their homework regularly. Find the following probabilities if a student is chosen at random:
    \begin{enumerate}
        \item \( P(\text{the student passes the course}) \)
        \item \( P(\text{the student did homework} | \text{the student passes the course}) \)
        \item \( P(\text{the student passes the course} | \text{the student did homework}) \)
    \end{enumerate}
\end{puzzle}

\begin{puzzle}
    A city's voter composition is 60\% Democrat and 40\% Republican. In the last mayoral election, 60\% of Democrats and 95\% of Republicans voted for their respective party's candidate. Determine which party's mayor runs city hall.
\end{puzzle}

\begin{puzzle}
    In a sample consisting of 500 people, 240 have $XY$ karyotype (chromosome type) and 259 have $XX$ karyotype and 1 has $XXY$ karyotype. Of these, 19 of the $XY$ karyotype and 1 of the $XX$ are color-blind. (See also Example \ref{example_independence_color_blindness} for links to more information on color-blindness.) Assuming the sample is representative of the population, calculate:
    \begin{enumerate}
        \item The percentage of the population that is color-blind.
        \item The probability that a person found to be color-blind has karyotype $XY$.
    \end{enumerate}
\end{puzzle}

\begin{puzzle}
    A test for a certain disease gives a positive result 95\% of the time if the person actually carries the disease. However, the test also gives a positive result 3\% of the time when the individual is not carrying the disease. It is known that 10\% of the population carries the disease. If the test is positive for a person, what is the probability that he or she has the disease?
\end{puzzle}

\begin{puzzle}
    A person has two coins: a fair coin and a two-headed coin. A coin is selected at random, and tossed. If the coin shows a head, what is the probability that the coin is fair?
\end{puzzle}

\begin{puzzle}
    A computer company buys its chips from three different manufacturers. Manufacturer I provides 60\% of the chips and is known to produce 5\% defective; Manufacturer II supplies 30\% of the chips and makes 4\% defective; while the rest are supplied by Manufacturer III with 3\% defective chips. If a chip is chosen at random, find the following probabilities:
    \begin{enumerate}
        \item \( P(\text{the chip is defective}) \)
        \item \( P(\text{it came from Manufacturer II } | \text{ the chip is defective}) \)
        \item \( P(\text{the chip is defective } | \text{ it came from manufacturer III}) \)
    \end{enumerate}
\end{puzzle}

\begin{puzzle}
    Lincoln Union High School District is made up of three high schools: Monterey, Fremont, and Kennedy, with an enrollment of 500, 300, and 200, respectively. On a given day, the percentage of students absent at Monterey High School is 6\%, at Fremont 4\%, and at Kennedy 5\%. If a student is seen at random, find the following probabilities. Hint: Convert the enrollments into percentages.
    \begin{enumerate}
        \item \( P(\text{the student is absent}) \)
        \item \( P(\text{the student came from Kennedy } | \text{ the student is absent}) \)
        \item \( P(\text{the student is absent } | \text{ the student came from Fremont}) \)
    \end{enumerate}
\end{puzzle}
