\section{Applications}
In this section, you will learn to use linear functions to model real-world applications.

Now that we have learned to determine equations of lines, we get to apply these ideas in a variety of real-life situations.  
Read the problem carefully. Highlight important information.  Keep track of which values correspond to the independent variable ($x$) and which correspond to the dependent variable ($y$).

\begin{example}
A taxi service charges \$0.50 per mile plus a \$5 flat fee.  What will be the cost of traveling 20 miles?  What will be cost of traveling $x$ miles?
\end{example}

\begin{solution}
Let $x$ be the distance traveled, in miles, and $y$ be the cost in dollars.

The cost of traveling 20 miles is $y = (0.50)(20) + 5 = 10 + 5 = 15$ dollars.

The cost of traveling $x$ miles is $y = (0.50)(x) + 5 = 0.50x + 5$ dollars.

In this problem, \$0.50 per mile is referred to as the variable cost, and the flat charge \$5 as the fixed cost.  Now if we look at our cost equation $y = 0.50x + 5$, we can see that the variable cost corresponds to the slope and the fixed cost to the $y$-intercept.
\end{solution}

\begin{example}
The variable cost to manufacture a product is \$10 per item and the fixed cost \$2500.  
If $x$ represents the number of items manufactured and $y$ represents the total cost, write the cost function.
\end{example}

\begin{solution}
The variable cost of \$10 per item tells us that $m = 10$.
The fixed cost represents the $y$-intercept, so $b = 2500$.
Therefore, the cost equation is $y = 10x + 2500$.
\end{solution}

\begin{example}
It costs \$750 to manufacture 25 items, and \$1000 to manufacture 50 items.  Assuming a linear relationship holds, find the cost equation, and use this function to predict the cost of 100 items.
\end{example}

\begin{solution}
Let $x$ be the number of items manufactured, and let $y$ be the cost.

Solving this problem is equivalent to finding an equation of a line that passes through the points $(25, 750)$ and $(50, 1000)$.  

$m = \frac{1000 - 750}{50 - 25} = 10$

Therefore, the partial equation is $y = 10x + b$.

By substituting one of the points in the equation, we get $b = 500$.

Therefore, the cost equation is $y = 10x + 500$.

To find the cost of 100 items, substitute $x = 100$ in the equation $y = 10x + 500$.

So the cost = $y = 10(100) + 500 = 1500$.

It costs \$1500 to manufacture 100 items.
\end{solution}

\begin{example}
The freezing temperature of water in Celsius is 0 degrees, and in Fahrenheit, it's 32 degrees. The boiling temperatures of water in Celsius and Fahrenheit are 100 degrees and 212 degrees, respectively. Write a conversion equation from Celsius to Fahrenheit and use this equation to convert 30 degrees Celsius into Fahrenheit.
\end{example}

\begin{solution}
Let's look at what is given:

\begin{center}
\begin{tabular}{|c|c|}
\hline
Celsius & Fahrenheit \\
\hline
0 & 32 \\
100 & 212 \\
\hline
\end{tabular}
\end{center}

Solving this problem is equivalent to finding an equation of a line that passes through the points $(0, 32)$ and $(100, 212)$. Since we are finding a linear relationship, we are looking for an equation $y = mx + b$, or in this case, $F = mC + b$, where $C$ represents the temperature in Celsius, and $F$ represents the temperature in Fahrenheit.

The slope $m = \frac{212 - 32}{100 - 0} = 95$.

The equation is $F = 95C + b$.

Substituting the point $(0, 32)$, we get $F = 95C + 32$.

To convert 30 degrees Celsius into Fahrenheit, substitute $C = 30$ in the equation:
\[F = 95C + 32\]
\[F = 95(30) + 32 = 86\]
\end{solution}

\begin{example}
The population of Canada in the year 1980 was 24.5 million, and in the year 2010, it was 34 million. The population of Canada over that time period can be approximately modeled by a linear function. Let $x$ represent time as the number of years after 1980, and let $y$ represent the size of the population.

a. Write the linear function that gives a relationship between the time and the population.

b. Assuming the population continues to grow linearly in the future, use this equation to predict the population of Canada in the year 2025.
\end{example}

\begin{solution}
The problem can be made easier by using 1980 as the base year, which means we choose the year 1980 as the year zero. This will make the year 2010 correspond to year 30. Now, let's look at the information we have:

\begin{center}
\begin{tabular}{|c|c|}
\hline
Year & Population \\
\hline
0 (1980) & 24.5 million \\
30 (2010) & 34 million \\
\hline
\end{tabular}
\end{center}
a. Solving this problem is equivalent to finding an equation of a line that passes through the points (0, 24.5) and (30, 34). We use these two points to find the slope:

\[ m = \frac{34 - 24.5}{30 - 0} = \frac{9.5}{30} = 0.32 \]

The y-intercept occurs when \( x = 0 \), so \( b = 24.5 \).

So, the equation relating time (\( x \)) and population (\( y \)) is:

\[ y = 0.32x + 24.5 \]

b. Now, to predict the population in the year 2025, we let \( x = 2025 - 1980 = 45 \):

\[ y = 0.32x + 24.5 \]

\[ y = 0.32(45) + 24.5 = 38.9 \]

In the year 2025, we predict that the population of Canada will be 38.9 million people.

Note that we assumed the population trend will continue to be linear. Therefore, if population trends change and this assumption does not continue to be true in the future, this prediction may not be accurate.
\end{solution}
