\section{Determining the Equation of a Line}

In this section, you will learn to:
\begin{enumerate}
    \item Find an equation of a line if a point and the slope are given.
    \item Find an equation of a line if two points are given.
\end{enumerate}

So far, we were given an equation of a line and were asked to give information about it. For example, we were asked to find points on the line, find its slope, and even find intercepts. Now we are going to reverse the process. That is, we will be given either two points or a point and the slope of a line, and we will be asked to find its equation.

An equation of a line can be written in three forms: the slope-intercept form, the point-slope form, or the standard form. We will discuss each of them in this section.

A line is completely determined by two points or by a point and slope. The information we are given about a particular line will influence which form of the equation is most convenient to use. Once we know any form of the equation of a line, it is easy to re-express the equation in the other forms if needed.

\textbf{The Slope-Intercept Form of a Line:} $y = mx + b$

In the last section, we learned that the equation of a line whose slope $= m$ and y-intercept $= b$ is $y = mx + b$. This is called the slope-intercept form of the line and is the most commonly used form.

\begin{example}
Find an equation of a line whose slope is $5$, and y-intercept is $3$.
\end{example}

\begin{solution}
Since the slope is $m = 5$, and the y-intercept is $b = 3$, the equation is $y = 5x + 3$.
\end{solution}

\begin{example}
Find the equation of the line that passes through the point $(2, 7)$ and has slope $3$.
\end{example}

\begin{solution}
Since $m = 3$, the partial equation is $y = 3x + b$. Now $b$ can be determined by substituting the point $(2, 7)$ in the equation $y = 3x + b$.

$7 = 3(2) + b$

$b = 1$

Therefore, the equation is $y = 3x + 1$.
\end{solution}

\begin{example}
Find an equation of the line that passes through the points $(-1, 2)$ and $(1, 8)$.
\end{example}

\begin{solution}
$m = \frac{8 - 2}{1 - (-1)} = \frac{6}{2} = 3.$

So the partial equation is $y = 3x + b$. We can use either of the two points $(-1, 2)$ or $(1, 8)$ to find $b$. Substituting $(-1, 2)$ gives

$2 = 3(-1) + b$

$5 = b$

So the equation is $y = 3x + 5$.
\end{solution}

\begin{example}
Find an equation of the line that has x-intercept $3$, and y-intercept $4$.
\end{example}

\begin{solution}
The x-intercept $= 3$, and y-intercept $= 4$ correspond to the points $(3, 0)$ and $(0, 4)$, respectively.

$m = \frac{4 - 0}{0 - 3} = \frac{-4}{3}$

We are told the y-intercept is $4$; thus $b = 4$.

Therefore, the equation is $y = -\frac{4}{3}x + 4$.
\end{solution}

\textbf{The Point-Slope Form of a Line:} $y - y1 = m(x - x1)$

The point-slope form is useful when we know two points on the line and want to find the equation of the line.

Let $L$ be a line with slope $m$, and known to contain a specific point $(x1, y1)$. If $(x, y)$ is any other point on the line $L$, then the definition of a slope leads us to the point-slope form or point-slope formula.

The slope is $\frac{y - y1}{x - x1} = m$

Multiplying both sides by $(x - x1)$ gives the point-slope form:

$y - y1 = m(x - x1)$

\begin{example}
Find the point-slope form of the equation of a line that has slope $1.5$ and passes through the point $(12,4)$.
\end{example}

\begin{solution}
Substituting the point $(x1, y1) = (12,4)$ and $m= 1.5$ in the point-slope formula, we get

$y - 4 = 1.5(x - 12)$

The student may be tempted to simplify this into the slope-intercept form $y = mx + b$. But since the problem specifically requests point-slope form, we will not simplify it.
\end{solution}

\textbf{The Standard Form of a Line:} $Ax + By = C$

Another useful form of the equation of a line is the standard form.

If we know the equation of a line in point-slope form, $y - y1 = m(x - x1)$, or if we know the equation of the line in slope-intercept form $y = mx + b$, we can simplify the formula to have all terms for the $x$ and $y$ variables on one side of the equation, and the constant on the other side of the equation.

The result is referred to as the standard form of the line: $Ax + By = C$.

\begin{example}
Using the point-slope formula, find the standard form of an equation of the line that passes through the point (2, 3) and has slope $-\frac{3}{5}$.
\end{example}

\begin{solution}
Substituting the point $(2, 3)$ and $m = -\frac{3}{5}$ in the point-slope formula, we get
\[y - 3 = -\frac{3}{5}(x - 2).\]

Multiplying both sides by 5 gives us
\[5(y - 3) = -3(x - 2),\]
\[5y - 15 = -3x + 6,\]
\[3x + 5y = 21 \text{ Standard Form}.\]
\end{solution}


\begin{example}
Find the standard form of the line that passes through the points (1, -2) and (4, 0).
\end{example}

\begin{solution}
First, we find the slope: $m = \frac{0 - (-2)}{4 - 1} = \frac{2}{3}.$

Then, the point-slope form is: $y - (-2) = \frac{2}{3}(x - 1).$

Multiplying both sides by 3 gives us
\[3(y + 2) = 2(x - 1),\]
\[3y + 6 = 2x - 2,\]
\[-2x + 3y = -8,\]
\[2x - 3y = 8 \text{ Standard Form}.\]
\end{solution}


\begin{example}
Write the equation $y = -\frac{2}{3}x + 3$ in the standard form.
\end{example}

\begin{solution}
Multiplying both sides of the equation by 3, we get
\[3y = -2x + 9,\]
\[2x + 3y = 9 \text{ Standard Form}.\]
\end{solution}


\begin{example}
Write the equation $3x - 4y = 10$ in the slope-intercept form.
\end{example}

\begin{solution}
Solving for $y$, we get
\[-4y = -3x + 10,\]
\[y = \frac{3}{4}x - \frac{5}{2} \text{ Slope Intercept Form}.\]
\end{solution}

\begin{example}
Find the slope of the following lines, by inspection.
\begin{enumerate}
    \item $3x - 5y = 10$
    \item $2x + 7y = 20$
    \item $4x - 3y = 8$
\end{enumerate}
\end{example}

\begin{solution}
\begin{enumerate}
    \item For $3x - 5y = 10,$ we have $A = 3$ and $B = -5,$ therefore, $m = -\frac{A}{B} = -\frac{3}{-5} = \frac{3}{5}.$
    
    \item For $2x + 7y = 20,$ we have $A = 2$ and $B = 7,$ therefore, $m = -\frac{A}{B} = -\frac{2}{7}.$
    
    \item For $4x - 3y = 8,$ we have $A = 4$ and $B = -3,$ therefore, $m = -\frac{A}{B} = -\frac{4}{-3} = \frac{4}{3}.$
\end{enumerate}
\end{solution}


\begin{example}
Find an equation of the line that passes through $(2, 3)$ and has slope $-\frac{4}{5}.$
\end{example}

\begin{solution}
Since the slope of the line is $-\frac{4}{5},$ we know that the left side of the equation is $4x + 5y,$ and the partial equation is going to be
\[4x + 5y = c.\]

Of course, $c$ can easily be found by substituting for $x$ and $y.$
\[4(2) + 5(3) = c,\]
\[8 + 15 = c,\]
\[23 = c.\]

The desired equation is
\[4x + 5y = 23.\]
\end{solution}

If you use this method often enough, you can do these problems very quickly.    
We summarize the forms for equations of a line below:
\begin{summarybox} Equations of Lines
\begin{itemize}
  \item Slope-Intercept form: $y = mx + b$, where $m$ is the slope and $b$ is the $y$-intercept.
  \item Point-Slope form: $y - y_1 = m(x - x_1)$, where $m$ is the slope and $(x_1, y_1)$ is a point on the line.
  \item Standard form: $Ax + By = C$.
  \item Horizontal Line: $y = b$, where $b$ is the $y$-intercept.
  \item Vertical Line: $x = a$, where $a$ is the $x$-intercept.
\end{itemize}
\end{summarybox}
