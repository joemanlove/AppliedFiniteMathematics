\chapter{Linear Equations} % Replace X with the chapter number and title

In this chapter, you will learn to:

\begin{enumerate}
    \item Graph a linear equation.
    \item Find the slope of a line.
    \item Determine an equation of a line.
    \item Solve linear systems.
    \item Do application problems using linear equations.
\end{enumerate}

\section{Graphing a Linear Equation}

In this section, you will learn to:

\begin{enumerate}
    \item Graph a line when you know its equation.
    \item Graph a line when you are given its equation in parametric form.
    \item Graph and find equations of vertical and horizontal lines.
\end{enumerate}

\subsection{Graphing a Line from its Equation}

Equations whose graphs are straight lines are called linear equations. The following are some examples of linear equations:

\begin{align*}
2x - 3y &= 6, \\
3x &= 4y - 7, \\
y &= 2x - 5, \\
2y &= 3, \\
x - 2 &= 0.
\end{align*}

A line is completely determined by two points. Therefore, to graph a linear equation, we need to find the coordinates of two points. This can be accomplished by choosing an arbitrary value for x or y and then solving for the other variable.

\begin{example}
Graph the line \(y = 3x + 2\).
\end{example}

\textbf{Solution:} We need to find the coordinates of at least two points. We arbitrarily choose \(x = -1\), \(x = 0\), and \(x = 1\).

If \(x = -1\), then \(y = 3(-1) + 2\) or \(y = -1\). Therefore, \((-1, -1)\) is a point on this line.

If \(x = 0\), then \(y = 3(0) + 2\) or \(y = 2\). Hence the point \((0, 2)\).

If \(x = 1\), then \(y = 5\), and we get the point \((1, 5)\). Below, the results are summarized, and the line is graphed.

\begin{center}
\begin{tabular}{c|c}
    \(x\) & \(y\) \\
    \hline
    -1 & -1 \\
    0  & 2  \\
    1  & 5
\end{tabular}
\end{center}

% TODO add graphic here


\begin{example}
Graph the line: $2x + y = 4$
\end{example}

\textbf{Solution:} Again, we need to find coordinates of at least two points. We arbitrarily choose $x = -1$, $x = 0$, and $y = 2$.

If $x = -1$, then $2(-1) + y = 4$ which results in $y = 6$. Therefore, $(-1, 6)$ is a point on this line.

If $x = 0$, then $2(0) + y = 4$, which results in $y = 4$. Hence the point $(0, 4)$.

If $y = 2$, then $2x + 2 = 4$, which yields $x = 1$, and gives the point $(1, 2)$. The table below shows the points, and the line is graphed.

% TODO add graphic here

\begin{center}
\begin{tabular}{c|c}
    $x$ & $y$ \\
    \hline
    $-1$ & $6$ \\
    $0$  & $4$ \\
    $1$  & $2$
\end{tabular}
\end{center}

\subsection{Intercepts:} The points at which a line crosses the coordinate axes are called the intercepts. When graphing a line by plotting two points, using the intercepts is often preferred because they are easy to find.
\begin{itemize}
    \item To find the value of the x-intercept, we let $y = 0$.
    \item To find the value of the y-intercept, we let $x = 0$.
\end{itemize}

\begin{example}
Find the intercepts of the line: $2x - 3y = 6$, and graph.
\end{example}

\textbf{Solution:} To find the x-intercept, let $y = 0$ in the equation, and solve for $x$.
\begin{align*}
2x - 3(0) &= 6 \\
2x &= 6 \\
x &= 3
\end{align*}
Therefore, the x-intercept is the point $(3, 0)$.

To find the y-intercept, let $x = 0$ in the equation, and solve for $y$.
\begin{align*}
2(0) - 3y &= 6 \\
0 - 3y &= 6 \\
-3y &= 6 \\
y &= -2
\end{align*}
Therefore, the y-intercept is the point $(0, -2)$.

To graph the line, plot the points for the x-intercept $(3, 0)$ and the y-intercept $(0, -2)$, and use them to draw the line.

% TODO add graphic here

\subsection{Graphing a Line from Its Equation in Parametric Form}

In higher math, equations of lines are sometimes written in parametric form. For example, $x = 3 + 2t, y = 1 + t$. The letter $t$ is called the parameter or the dummy variable.

Parametric lines can be graphed by finding values for $x$ and $y$ by substituting numerical values for $t$. Plot the points using their $(x, y)$ coordinates and use the points to draw the line.

\begin{example}
Graph the line given by the parametric equations: $x = 3 + 2t$, $y = 1 + t$
\end{example}

\textbf{Solution:} Let $t = 0, 1$ and $2$; for each value of $t$, find the corresponding values for $x$ and $y$. The results are given in the table below.

\begin{center}
\begin{tabular}{c|c|c}
    $t$ & $x$ & $y$ \\
    \hline
    $0$ & $3$ & $1$ \\
    $1$ & $5$ & $2$ \\
    $2$ & $7$ & $3$ \\
\end{tabular}
\end{center}

% TODO add graphic here

\subsection{Horizontal and Vertical Lines}

When an equation of a line has only one variable, the resulting graph is a horizontal or a vertical line.

The graph of the line \(x = a\), where \(a\) is a constant, is a vertical line that passes through the point \((a, 0)\). Every point on this line has the \(x\)-coordinate equal to \(a\), regardless of the \(y\)-coordinate.

The graph of the line \(y = b\), where \(b\) is a constant, is a horizontal line that passes through the point \((0, b)\). Every point on this line has the \(y\)-coordinate equal to \(b\), regardless of the \(x\)-coordinate.

\begin{example}
Graph the lines: \(x = -2\), and \(y = 3\).
\end{example}

\textbf{Solution:} The graph of the line \(x = -2\) is a vertical line that has the \(x\)-coordinate \(-2\) no matter what the \(y\)-coordinate is. The graph is a vertical line passing through point \((-2, 0)\).

The graph of the line \(y = 3\) is a horizontal line that has the \(y\)-coordinate \(3\) regardless of what the \(x\)-coordinate is. Therefore, the graph is a horizontal line that passes through point \((0, 3)\).

% TODO add graphic here


\section{Slope of a Line}

In this section, you will learn to:
\begin{enumerate}
    \item Find the slope of a line.
    \item Graph the line if a point and the slope are given.
\end{enumerate}

In the last section, we learned to graph a line by choosing two points on the line. A graph of a line can also be determined if one point and the "steepness" of the line is known. The number that refers to the steepness or inclination of a line is called the slope of the line. From previous math courses, many of you remember slope as the "rise over run," or "the vertical change over the horizontal change" and have often seen it expressed as:
\[
\text{slope} = \frac{{y_2 - y_1}}{{x_2 - x_1}}
\]
We give a precise definition.

\begin{definition} If \((x_1, y_1)\) and \((x_2, y_2)\) are two different points on a line, the slope of the line is
\[
\text{slope} = m = \frac{{y_2 - y_1}}{{x_2 - x_1}}
\]
\end{definition}

\begin{example}
Find the slope of the line passing through points \((-2, 3)\) and \((4, -1)\), and graph the line.
\end{example}

\textbf{Solution:} Let \((x_1, y_1) = (-2, 3)\) and \((x_2, y_2) = (4, -1)\), then the slope is

\[
\text{slope} = m = \frac{{-1 - 3}}{{4 - (-2)}}
\]

To give the reader a better understanding, both the vertical change, \(-4\), and the horizontal change, \(6\), are shown in the above figure.

When two points are given, it does not matter which point is denoted as \((x_1, y_1)\) and which \((x_2, y_2)\). The value for the slope will be the same.

%TODO Fix Example numbering
In Example 1, if we instead choose \((x_1, y_1) = (4, -1)\) and \((x_2, y_2) = (-2, 3)\), then we will get the same value for the slope as we obtained earlier.

The steps involved are as follows:
\[
\text{m} = \frac{3 - (-1)}{-2 - 4} = \frac{4}{-6} = -\frac{2}{3}
\]

The student should further observe that
\begin{itemize}
    \item If a line rises when going from left to right, then it has a positive slope. In this situation, as the value of $x$ increases, the value of $y$ also increases.
    \item If a line falls going from left to right, it has a negative slope; as the value of $x$ increases, the value of $y$ decreases.
\end{itemize}

\begin{example}
Find the slope of the line that passes through the points $(2, 3)$ and $(2, -1)$, and graph.
\end{example}

\textbf{Solution:} Let $(x_1, y_1) = (2, 3)$ and $(x_2, y_2) = (2, -1)$, then the slope is
\[ m = \frac{-1 - 3}{2 - 2} = \frac{-4}{0} = \text{undefined}.\]

% TODO add graphic here

\begin{note}
The slope of a vertical line is undefined.
\end{note}

\begin{example}
Find the slope of the line that passes through the points $(-1, -4)$ and $(3, -4)$.
\end{example}

\textbf{Solution:} Let $(x_1, y_1) = (-1, -4)$ and $(x_2, y_2) = (3, -4)$, then the slope is
\[ m = \frac{-4 - (-4)}{3 - (-1)} = \frac{0}{4} = 0.\]

% TODO add graphic here

\begin{note}
The slope of a horizontal line is $0$.
\end{note}

\begin{example}
Graph the line that passes through the point $(1, 2)$ and has a slope of $-\frac{3}{4}$.
\end{example}

\textbf{Solution:} The slope equals $\frac{\text{rise}}{\text{run}}$. The fact that the slope is $\frac{-3}{4}$ means that for every rise of $-3$ units (fall of 3 units), there is a run of 4 units. So if from the given point $(1, 2)$ we go down 3 units and go right 4 units, we reach the point $(5, -1)$. The graph is obtained by connecting these two points.

% TODO add graphic here

Alternatively, since $\frac{3}{-4}$ represents the same number, the line can be drawn by starting at the point $(1, 2)$ and choosing a rise of 3 units followed by a run of $-4$ units. So from the point $(1, 2)$, we go up 3 units and to the left 4 units, thus reaching the point $(-3, 5)$, which is also on the same line. See figure below.

% TODO add graphic here

\begin{example}
Find the slope of the line $2x + 3y = 6$.	
\end{example}

\textbf{Solution:} In order to find the slope of this line, we will choose any two points on this line.
Again, the selection of $x$ and $y$ intercepts seems to be a good choice. The $x$-intercept is $(3, 0)$, and the $y$-intercept is $(0, 2)$. Therefore, the slope is
\[ m = \frac{2 - 0}{3 - 0} = \frac{-2}{3} .\]
The graph below shows the line and the $x$-intercepts and $y$-intercepts: 

% TODO add graphic here

\begin{example}
Find the slope of the line $y = 3x + 2$.
\end{example}

\textbf{Solution:} We again find two points on the line, say $(0, 2)$ and $(1, 5)$.
Therefore, the slope is
\[ m = \frac{5 - 2}{1 - 0} = \frac{3}{1} = 3.\]
Look at the slopes and the $y$-intercepts of the following lines.
\begin{center}

\begin{tabular}{ccc}
Line & Slope & Y-Intercept \\
$y = 3x + 2$ & $3$ & $2$ \\
$y = -2x + 5$ & $-2$ & $5$ \\
$y = \frac{3}{2}x - 4$ & $\frac{3}{2}$ & $-4$
\end{tabular}
\end{center}

It is no coincidence that when an equation of the line is solved for $y$, the coefficient of the $x$ term represents the slope, and the constant term represents the $y$-intercept.
In other words, for the line $y = mx + b$, $m$ is the slope, and $b$ is the $y$-intercept.

\begin{example}
Determine the slope and $y$-intercept of the line $2x + 3y = 6$.
\end{example}

\textbf{Solution:} We solve for $y$:
\[2x + 3y = 6\]
\[3y = -2x + 6\]
\[y = -\frac{2}{3}x + 2\]
The slope is equal to the coefficient of the $x$ term, which is $-\frac{2}{3}$.
The $y$-intercept is equal to the constant term, which is $2$.
